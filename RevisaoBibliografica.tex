% ----------------------------------------------------------
% Introdução (exemplo de capítulo sem numeração, mas presente no Sumário)
% ----------------------------------------------------------
\chapter[Revisão Bibliográfica]{Revisão Bibliográfica}
%$\addcontentsline{toc}{chapter}{Introdução}
% ----------------------------------------------------------

A literatura sobre Receptores Ópticos \'e extensa e contempla os mais diversos aspectos sobre o tema, desde o entendimento do fenômeno físico, diferentes topologias e estrat\'egias para otimização de diferentes características do circuito, circuitos auxiliares necessários para a transferência de informação, e tamb\'em o layout do projeto. Este capítulo tem como objetivo contextualizar e fundamentar algumas discussões de capítulos posteriores, demonstrar alguns trabalhos previamente realizados na área, e dar uma perspectiva histórica sobre algumas tecnologias envolvidas.
\section{Histórico}
O interesse de extrair informações advindas do espectro eletromagn\'etico \'e a muito tempo de grande interesse para físicos e engenheiros. As suas aplicações são diversas, desde extrair cores para formar uma imagem at\'e extrair informação digital de acordo com um sinal luminoso modulado.
Em 1949, a primeira patente relacionada a um dispositivo semicondutor capaz de transformar informações luminosas em el\'etricas foi requisitada pela Bell Labs, sendo oficialmente aceita em 1951. Era o nascimento do fototransistor, um transistor bipolar especial que capta informação luminosa em sua base, gerando uma informação por forma de uma corrente el\'etrica em seu coletor e emissor \cite{Shive}.

Com a invenção do fototransistor e fotodiodo, e o desenvolvimento de tecnologias de fabricação de semicondutores, se gerou uma intensa pesquisa na d\'ecada de 60, que culminou no desenvolvimento do CCD (\textit{Charge Coupled Device}) no início da d\'ecada de 70, sendo a tecnologia de Recepção Óptica mais utilizada at\'e a d\'ecada de 90 \cite{EstevaoCoelho, Andre}.

Em meados da d\'ecada de 70, o PPS (\textit{Passive Pixel Sensor}) foi desenvolvido, como alternativa aos dispositivos de imagem de tubo a vácuo \cite{Savvas}. Sua constituição se dá por uma matriz de fotodiodos, que convertem uma informação luminosa em el\'etrica, diretamente a um elemento processador de dados analógico, sem a utilização de um componente de amplificação.

Ainda na d\'ecada de 70, os primeiros trabalhos relacionados ao APS foram desenvolvidos utilizando a tecnologia MOS (\textit{Metal-Oxide-Semiconductor}) \cite{Peter}. O APS \'e uma evolução direta do PPS, com a adição de circuitos eletrônicos de condicionamento de sinal, utilizando-se amplificadores operacionais \cite{EstevaoCoelho}. Amplificar o sinal permite uma maior precisão na aquisição dos dados, al\'em de permitir uma maior miniaturização do fotodiodo, o que leva a uma miniaturização de todo o sistema. Com a evolução da tecnologia CMOS (\textit{Complementary Metal-Oxide-Semiconductor}) na d\'ecada de 90, o APS se tornou mais atraente, por ser facilmente integrado com outros componentes que utilizam-se da mesma tecnologia em um mesmo circuito integrado.

Amplificadores Operacionais são circuitos amplamente utilizados para o desenvolvimento de sistemas analógicos. Desde a década de 60, o desenvolvimento de circuitos de comunicação óptica exploram a utilização de amplificadores para captação de sinais ópticos, transmissão e regeneração de sinais. A utilização de amplificadores operacionais junto à fotodiodos, possibilitam a conversão de sinais de luz em sinais elétricos, que pode ser processado e utilizado para fins tanto analógicos quanto digitais. O circuito comumente utilizado no processo de captação óptica de sinais digitais são variações do TIA, tendo papel fundamental para a expansão tecnológica que temos desde então. \cite{ajoy, andrefontoura}.

\section{Fotodiodo}

O fotodiodo, ou fotodetector, \'e um componente optoeletrônico semicondutor que tem a função de captar um sinal eletromagn\'etico no ambiente, e então convertê-lo para uma corrente el\'etrica, chamada de corrente fotogerada, ou fotocorrente \cite{RazaviOpt}. A faixa de frequência da onda eletromagnética que pode ser captada varia em cada projeto, mas dispositivos práticos se apresentam comumente em uma banda entre o ultravioleta e o infravermelho, o que possibilita o desenvolvimento de dispositivos dentro da faixa da luz visível \cite{LidianeCampos}.

Para o entendimento do funcionamento do fotodiodo, uma fundamentação quântica da luz deve ser utilizada, com foco principal na Teoria das Bandas e no entendimento do que são os Fótons.

\subsection{Física Quântica e Fótons}
A Física Quântica nos diz que qualquer material apresenta níveis de energia possíveis dados de forma discreta (quantizada), ou seja, qualquer objeto do universo não pode ter qualquer nível de energia, mas sim valores dados de forma discreta, e que são quantificados por meio da solução da \textit{Equação de Schrödinger} do material \cite{Sze, JohnSingleton}.


\subsubsection{Teoria das Bandas}

A Teoria das Bandas nasce da observação de que comumente uma grande quantidade de níveis de energia permitidos de um material se apresentam muito próximos entre si. Desses níveis de energia, surge-se o conceito “banda de energia”, que \'e a aproximação que dentro dessa faixa de níveis concentrados, temos uma representação contínua de níveis de energia permitidos. Um material pode apresentar uma grande quantidade de bandas, sendo que duas são de extremo interesse para o estudo de semicondutores: A Banda de condução e a Banda de valência.

A Banda de valência \'e a última da qual se considera que um el\'etron está fortemente ligada a um átomo, não podendo mover livremente ao longo do material. A Banda de condução \'e a primeira banda de níveis acima da de Valência, e nele já consideramos que um el\'etron \'e livre para se movimentar no material.

\subsubsection{Fóton}

Fóton \'e uma partícula elementar que surge como resultado da liberação de energia ocasionada pela migração de um el\'etron da Banda de condução para o de valência. Este \'e o elemento fundamental constituinte da luz, e pela dualidade onda-partícula, pode ser visto tanto do ponto de vista de uma partícula, que se movimenta no espaço e pode chocar com um material; quanto de onda movimentando no espaço, apresentando uma frequência formada pela oscilação de dois campos (El\'etrico e Magn\'etico), que \'e regida pelas \textit{Leis de Maxwell} \cite{Sze, JohnSingleton}.

\subsection{Fotodiodo}
\label{secao_fotodiodo}
Um fotodiodo \'e formado por uma junção PN, que \'e a junção de um material semicondutor dopado do tipo-P com um do tipo-N, como mostrado na \autoref{fig_fotodiodo}:

\begin{figure}[!h]
	\caption{\label{fig_fotodiodo}Exemplo de construção de uma junção PN}
	\begin{center}
	    \includegraphics[scale=0.7]{Imagens/JuncaoPN.png}
	\end{center}
	\legend{Fonte: \cite{LidianeCampos}}
\end{figure}

Quando um fóton com energia suficiente para excitar o material \'e absorvido, um par el\'etron-lacuna \'e gerado (migração de um el\'etron da banda de valência para de condução), e por difusão, os el\'etrons migram para o cátodo, formando uma fotocorrente. Ajustando-se os tamanhos das camadas P, N e também a dopagem, podemos controlar a resposta em frequência do fotodiodo e qual a energia mínima do fóton necessária para que gere a fotocorrente.

A Região de Depleção é uma região de barreira de potencial formada ao redor dos contatos da junção, que é gerada com a difusão de elétrons da camada N para P já na construção do dispositivo, ocorrendo devido a diferença de concentração de elétrons em N e lacunas (espaços com ausência de elétrons mas que podem ser ocupados pelos mesmos). No fotodiodo essa é a região de maior interesse, por ser onde há a conversão de fótons em energia elétrica que pode ser aproveitada.

Com a migração dos el\'etrons, uma diferença de potencial \'e gerada entre as camadas P e N, que pode ser aproveitada para a geração de corrente el\'etrica em um circuito el\'etrico fechado \cite{hamamatsu}.
Um fotodiodo apresenta um modelo el\'etrico equivalente ilustrado na \autoref{fig_modelofotodiodo}:

\begin{figure}[!h]
	\caption{\label{fig_modelofotodiodo}Modelo el\'etrico de um fotodiodo}
	\begin{center}
	    \includegraphics[scale=0.5]{Imagens/ModeloFotodiodo.png}
	\end{center}
	\legend{Fonte: \cite{hamamatsu}}
	\label{modeloElFotodiodo}
\end{figure}

    O modelo el\'etrico apresenta a seguinte expressão:

\begin{equation}
    \label{eq_modEletFot}
    I_o = I_L - I_D - I\rq = I_L - I_S*(\exp (\frac{qV_D}{kT})-1) - I\rq
\end{equation}

Onde:
\begin{itemize}
    \item \textit{I\textsubscript{o}} \'e a corrente de saída presente na carga [\textit{A}]
    \item \textit{V\textsubscript{o}} \'e a tensão de saída [\textit{V}]
    \item \textit{R\textsubscript{L}} \'e a carga de saída [\textit{$\Omega$}]
    \item \textit{V\textsubscript{D}} \'e a tensão presente no diodo do modelo el\'etrico equivalente [\textit{V}]
    \item $I_L$ \'e a corrente fotogerada pela fonte luminosa[\textit{A}]
    \item  \textit{I\textsubscript{D}} \'e a corrente de escuro do fotodiodo (sem a presença de luz) [\textit{A}]
    \item \textit{C\textsubscript{j}} \'e a capacit\^ancia de junção [\textit{F}]
    \item \textit{R\textsubscript{sh}} \'e a resist\^encia shunt do modelo el\'etrico equivalente [$\Omega$]
    \item \textit{I\rq} \'e a corrente shunt presente na resistência shunt do modelo el\'etrico equivalente [\textit{A}]
    \item \textit{R\textsubscript{S}} \'e a resistência em s\'erie com a carga de saída do modelo el\'etrico equivalente [$\Omega$]
    \item \textit{q} \'e a carga el\'etrica de um el\'etron [\textit{C}],
    \item \textit{k} \'e a constante de Boltzmann [\textit{J.K$^{-1}$}],
    \item \textit{T} \'e a temperatura presente no fotodiodo [\textit{K}]
\end{itemize}

O fotodiodo em uma grande faixa apresenta comportamento linear em relação a quantidade de corrente fotogerada e o número de fótons recebidos, sendo essa a faixa de principal interesse em aplicações como o APS ou o TIA. Além disso, a corrente fotogerada e a queda de tensão entre os terminais do fotodiodo também tendem a apresentar resposta linear, como mostrado na \autoref{fig_respFotodiodo}. Com o aumento ou diminuição expressiva, o circuito começa a se tornar não-linear.

Nota-se na \autoref{fig_respFotodiodo} que uma corrente e tensão negativa apresenta uma maior faixa de operação antes que ocorra a saturação do circuito. Isso ocorre pois a polarização reversa do fotodiodo aumenta a região de depleção, aprimorando a captação de fótons que podem ser convertidos em um sinal elétrico. Devido a isso o fotodiodo é comumente polarizado reversamente nas suas diversas aplicações.

\begin{figure}[!h]
	\caption{\label{fig_respFotodiodo}Resposta Tensão x Corrente para dada intensidade luminosa no fotodiodo}
	\begin{center}
	    \includegraphics[scale=0.8]{Imagens/graficoRespostaFotodiodo.png}
	\end{center}
	\legend{Fonte: \cite{hamamatsu}}
\end{figure}

\subsection{Figuras de M\'erito}
Do fotodiodo podemos extrair diversas m\'etricas (Figuras de M\'erito), que são de interesse para comparar diferentes modelos. As principais Figuras de M\'erito encontradas na literatura são descritas abaixo, apresentadas em \cite{LidianeCampos}.

\subsubsection{Eficiência Quântica}
Relação entre o número de portadores detectados nos terminais das camadas PN do fotodetector, dividido pela incidência de uma determinada quantidade de fótons no fotodetector.

\begin{equation}
    \eta = \frac{N_e}{N_p}
\end{equation}

Onde:
\begin{itemize}
    \item \textit{$\eta$} \'e Efici\^encia Qu\^antica [\textit{Adm.}]
    \item \textit{N$_e$} \'e o n\'umero de portadores que podem ser detectados nos terminais externos do fotodetector [\textit{n° portadores}]
    \item \textit{N$_p$} incid\^encia de determinada quantidade de f\'otons [\textit{n° f\'otons}]
\end{itemize}

\subsubsection{Responsividade}
Razão entre a corrente fotogerada e a potência óptica incidida no fotodiodo.

\begin{equation}
    \label{eq_responsividade}
    R_\lambda = \frac{I_{PH}}{P_{FD}}
\end{equation}

Onde:
\begin{itemize}
    \item \textit{R$_\lambda$} \'e a Responsividade [\textit{A.W$^{-1}$}]
    \item \textit{I$_{PH}$} \'e a corrente fotogerada [\textit{A}]
    \item \textit{P$_{FD}$} \'e a potência \'optica presente no fotodiodo [\textit{W}]
\end{itemize}

A Eficiência Quântica e a Responsividade se relacionam de acordo com a \autoref{eqEfResp}.

\begin{equation}
    \label{eqEfResp}
    R_\lambda = \frac{\lambda\eta}{1,24}
\end{equation}

Onde:
\begin{itemize}
    \item \textit{R$_\lambda$} \'e a Responsividade [\textit{A.W$^{-1}$}]
    \item $\lambda$ \'e o comprimento de onda da luz incidente [\textit{m}]
    \item $\eta$ \'e Efici\^encia Qu\^antica [\textit{Adm.}]
\end{itemize}

A \autoref{fig_eqEfResp} mostra o gráfico da Responsividade e Efici\^encia Qu\^antica de alguns materiais.

\begin{figure}[!h]
	\caption{\label{fig_responsividade}Responsividade e Efici\^encia Qu\^antica dos materiais Ge, InGaAs, Si}
	\begin{center}
	    \includegraphics[scale=0.5]{Imagens/GraficoRespostaEspectral.png}
	\end{center}
	\legend{Fonte: \cite{ajoy}}
	\label{fig_eqEfResp}
\end{figure}


\subsubsection{Velocidade de Resposta}
A velocidade de resposta indica o quão rápido o fotodiodo \'e capaz de responder a estímulos de luz externos, em determinada frequência. É caracterizado pelo Tempo de Subida e Tempo de Descida do Fotodiodo, na faixa de frequências de interesse.

O Tempo de Subida \'e calculado como o tempo do qual um fotodiodo, inicialmente sem incidência de luz, leva para elevar o seu nível de tensão nos seus terminais de 10\% para 90\% do seu pico, a partir do no momento que começar a absorver fótons de uma fonte luminosa controlada, em determinada frequência.

O Tempo de Descida \'e calculado como o tempo do qual um fotodiodo, inicialmente com determinada incidência de luz e já estabilizado na sua respectiva tensão de pico, leva para diminuir a diferença de tensão entre seus terminais de 90\% para 10\% do seu pico, a partir do momento que não absorve mais fótons de origem de uma fonte luminosa controlada, em determinada frequência.

\begin{figure}[!h]
	\caption{\label{fig_velocidadeResp}Representação gr\'afica do Tempo de Subida e Descida}
	\begin{center}
	    \includegraphics[scale=0.3]{Imagens/GraficoVelocidadeResposta.png}
	\end{center}
	\legend{Fonte: Produzido pelo autor}
\end{figure}

O conhecimento da velocidade de resposta \'e crucial para determinar as limitações do projeto em termos de processamento de informações. A amostragem de dados deve ter período maior do que os tempos de subida e de descida do fotodiodo, nos maiores valores apresentados na faixa de frequência desejada.

\subsubsection{Resposta Espectral}

O Efeito Fotoel\'etrico se caracteriza pela excitação de el\'etrons em um material por um fóton, quando este apresenta uma energia mínima, chamada de \textit{Função Trabalho}. Mesmo quando uma grande quantidade de fótons com energia menor do que a Função Trabalho atravessem o fotodiodo, eles não serão capazes de excitar nenhum el\'etron. Como a energia de um fóton diminui com o aumento de seu comprimento de onda, existe um comprimento de onda máximo (ou uma frequência mínima) do qual o fotodetector será capaz de gerar fotocorrente.

Por outro lado, uma diminuição do comprimento de onda tende a excitar cada vez menos el\'etrons, pois absorção de luz acontece principalmente na superfície da região de difusão, que não é a região de depleção.

Sabendo destas limitações, definimos a Resposta Espectral como a faixa de comprimentos de onda do qual o fotodetector \'e capaz de produzir uma fotocorrente correspondente.

\section{Sensor de Pixel Ativo (APS)}
\label{section:APS}
Um APS \'e um dispositivo do qual se aproveita das características de um fotodiodo para gerar um sinal que pode ser amostrado e então quantificado, de forma a produzir informações referentes à luz incidida no fotodetector.

Dentre as várias possibilidades de produção de um circuito APS, aquele estudado ao longo de todo este trabalho se apresenta na \autoref{fig_APS}.

\begin{figure}[!h]
	\caption{\label{fig_APS}Circuito APS do trabalho}
	\begin{center}
	    \includegraphics[scale=0.3]{Circuitos/APS.png}
	\end{center}
	\legend{Fonte: Produzido pelo autor}
\end{figure}

Cada componente desempenha uma diferente função de forma a processar a informação advinda da corrente fotogerada:

\begin{itemize}

    \item O transistor $T_{buffer}$ funciona como um amplificador de Dreno Comum, e seu papel \'e replicar o sinal advindo do n\'o central ($V_{cn}$) ao n\'o de sa\'ida ($V_{out}$), decrescido da diferença de tensão entre porta e dreno do transistor ($V_{GS}$). O efeito de carga\footnote{Efeito do qual a saída de um circuito 1, conectado a entrada de outro circuito 2, tem o valor de saída reduzido, comparado a situação da qual a saída do circuito 1 não está conectada em nada. Ocorre devido a saída do circuito 1 enxergar uma impedância finita na conexão do circuito 2, que então cria um divisor de tensão com a sua impedância de saída e consequentemente reduz a tensão vista no nó de saída.} que aconteceria caso o n\'o central fosse a saída do sistema é reduzido, pois entre os pinos de porta e dreno do transistor há a presença de uma alta impedância.

    \item O transistor $T_{reset}$ quando fechado (nível lógico '0' em seu gate) faz com que o potencial do nó $V_{cn}$ seja igual a VDD. Quando aberto (nível lógico '1' em seu gate), o sinal no n\'o central passa a depender da configuração de $T_{enable}$.

    \item A Porta de Transmissão $T_{enable}$ funciona como uma chave, e quando o transistor $T_{reset}$ estiver em aberto tem a função de isolar o fotodiodo do nó central de acordo com a sua situação de abertura ou fechamento. Quando aberto, $V_{pn}$ fica isolado do restante do circuito, devido ao estado de alta imped\^ancia nos terminais do dispositivo. Quando fechado, a corrente fotogerada descarrega em $V_{cn}$, pois a associação das capacitâncias parasitas presentes nos transistores de $T_{enable}$, $T_{reset}$ e $T_{buffer}$ formam um caminho fechado do qual descarrega lentamente o fotodiodo \cite{LidianeCampos}. Uma representação desses capacitores \'e dada na \autoref{fig_APS_cap}.
    
    \item O transistor $T_{select}$ \'e utilizado para que m\'ultiplos APS's compartilhem um mesmo barramento de sa\'ida. No projeto aqui implementado $T_{select}$ sempre se apresentar\'a fechado (\textit{SELECT} em n\'ivel l\'ogico '1').
    
    \item A fonte de corrente \textit{$I_{ref}$} tem a função de polarizar a sa\'ida do circuito, além de aprimorar a linearidade do estágio de sa\'ida \cite{RazaviFundM}.

\end{itemize}

    H\'a dois n\'os internos de bastante interesse ao trabalho, que são:

\begin{itemize}
    \item \textit{$V_{pn}$}: tensão entre os terminais do fotodiodo [\textit{V}]
    \item \textit{$V_{cn}$}: tensão no n\'o central do APS [\textit{V}]
\end{itemize}

    Os sinais externos do circuito são:
    
\begin{itemize}
    \item \textit{RESET}: fecha o transistor $T_{reset}$. Ativo em n\'ivel l\'ogico '0'.
     \item \textit{SELECT}: fecha o transistor $T_{select}$. Ativo em n\'ivel l\'ogico '1'. No circuito do trabalho estar\'a sempre configurado como '1'.
     \item \textit{ENABLE}: fecha o transmition gate $T_{enable}$. Ativo em n\'ivel l\'ogico '1'.
     \item \textit{VDD}: Alimentação do circuito
\end{itemize}

\subsection{Capacit\^ancias Parasitas}
Para o correto estudo do APS devemos observar a capacit\^ancia equivalente $C_{eq}$ presente nos n\'os $V_{pn}$ e $V_{cn}$ do circuito \cite{LidianeCampos}. Na \autoref{fig_APS_cap} temos uma representação das capacit\^ancias que compõem $C_{eq}$, que é a soma das capacitâncias apresentadas na figura.

\begin{figure}[!h]
	\caption{\label{fig_APS_cap}Representação do circuito APS com suas capacit\^ancias parasitas destacadas}
	\begin{center}
	    \includegraphics[scale=0.3]{Circuitos/APS_cap.png}
	\end{center}
	\legend{Fonte: Produzido pelo autor}
\end{figure}

Onde: 

\begin{itemize}
    \item $C_j$ \'e a capacit\^ancia de junção do fotodiodo. Seu principal efeito no estudo do APS \'e limitar a velocidade de variação da corrente fotogerada.
    
    \item $C_{cn}$ \'e a capacit\^ancia equivalente vista do n\'o $V_{cn}$ ao GND, devido principalmente às capacit\^ancias presentes em $T_{enable}$,  $T_{reset}$ e $T_{buffer}$, sendo o de $T_{buffer}$ o de maior contribuição. O capacitor cria um caminho fechado para a corrente do fotodiodo circular ao GND e ser descarregado em um dos est\'agios citados posteriormente no trabalho.
    
    \item $I_{ph}$ \'e a corrente fotogerada [\textit{A}].
    
    \item \textit{Diodo} \'e o diodo do modelo el\'etrico equivalente do fotodiodo.
\end{itemize}

\subsection{Est\'agios}
\label{estagiosAPS}

A operação do APS pode ser dividida em 4 est\'agios, que vão definir os m\'inimos momentos de  atuação no sinal de controle e seus limites operação. Os est\'agios são representados na \autoref{figura_estagiosAPS}, tendo como base o trabalho de \cite{LidianeCampos}.

\begin{figure}[!h]
	\caption{\label{figura_estagiosAPS}Esboço de resposta do APS ao controlar seus sinais de entrada}
	\begin{center}
	    \includegraphics[scale=0.2]{Imagens/estagiosAPS.png}
	\end{center}
	\legend{Fonte: Adaptado de \cite{LidianeCampos}}
\end{figure}

\begin{enumerate}

\item $T_{reset}$ fechado, $T_{enable}$ fechado (Período de Reset)
    
Nessa condição, o n\'o $V_{cn}$ \'e igual a VDD.

\item $T_{reset}$ aberto, $T_{enable}$ fechado (Período de Integração)

A corrente fotogerada passa a descarregar a capacit\^ancia $C_{cn}$ do nó central. A tensão no fotodiodo passa a diminuir, devido a circulação de corrente que diminui a carga de $C_{j}$.

A tensão $V_{cn}$ neste segundo est\'agio tem uma relação linear com a corrente fotogerada, conforme descrito na \autoref{eq_modEletFotIl}, que \'e deduzida fazendo an\'alises dos n\'os, desprezando-se a resistência presente em $T_{enable}$ e também outras resistências parasitas, como aquelas presentes no modelo elétrico do fotodiodo apresentado na \autoref{modeloElFotodiodo}.

\begin{equation}
    \label{eq_modEletFotIl}
    V_{cn}(t) = V_0-\frac{I_{PH}}{C_j+C_{cn}}t
\end{equation}

Onde:

\begin{itemize}
    \item $V_{cn}(t)$ \'e a tensão do nó $V_{cn}$ em determinado tempo $t$ [$V$]
    \item $V_0$ \'e a tensão do nó $V_{cn}$ no momento que o estágio inicia. Comumente deseja-se que esta tensão seja a máxima possível [$V$]
    \item $I_{PH}$ \'e a corrente fotogerada [$A$]
    \item $C_j$ \'e a capacit\^ancia de junção do fotodiodo [$F$]
    \item $C_{cn}$ \'e a capacit\^ancia do n\'o central do APS [$F$]
    \item $t$ \'e o tempo a partir do qual o estágio se iniciou [$s$]
\end{itemize}

Já a tensão de saída $V_{out}$ é dada pela \autoref{eq_voutaps}, que junto a \autoref{eq_modEletFotIl} resulta na \autoref{eq_voutfinal}, utilizada no trabalho para calcular a tensão de saída do APS em determinado instante de tempo do Estágio 2.

\begin{equation}
    \label{eq_voutaps}
    V_{out}(t) = V_{cn} - V_{GS\_buffer}
\end{equation}

\begin{equation}
    \label{eq_voutfinal}
    V_{out}(t) = V_0-\frac{I_{PH}}{C_j+C_{cn}}t - V_{GS\_buffer}
\end{equation}

Onde:

\begin{itemize}
    \item $V_{out}(t)$ \'e a tensão de saída do APS em determinado tempo $t$ [$V$]
    \item $V_{GS\_buffer}$ \'e a diferença de potencial entre $V_{cn}$ e $V_{out}(t)$, que considerando-se a ausência de efeitos de carga, e que $T_{buffer}$ esteja sempre na região de saturação no Estágio 2, é uma constante [$V$]
\end{itemize}

\item $T_{reset}$ aberto, $T_{enable}$ aberto, potencial \textit{$V_{pd}$} positivo

Logo ap\'os abrir o $T_{enable}$, a fotocorrente não circula mais no n\'o central, e o fotodiodo continua a ter sua diferença de potencial reduzida, com circulação de corrente internamente devido a $C_j$. O n\'o central passa a reduzir a tensão bem lentamente devido a correntes de fuga. \textit{$V_{pd}$} diminue seu valor at\'e se tornar negativo.

\item \textit{$T_{reset}$} aberto, \textit{$T_{enable}$} aberto, potencial \textit{$V_{pd}$} negativo

Nessa situação, o $T_{enable}$ passa a operar em modo linear. O potencial no n\'o $V_{cn}$ passa a diminuir devido a circulação de corrente do n\'o at\'e o fotodiodo. O nó diminui at\'e que finalmente chega a 0, onde se mant\'em estável e não apresenta mais circulação de corrente.

\end{enumerate}

\subsection{Amostrando informações da luz com um APS básico}
\label{secao_amostrando}

Podemos aproveitar o entendimento das propriedades f\'isicas do fotodiodo, e tamb\'em dos est\'agios de funcionamento de um APS, para obtermos informações relativas aos f\'otons absorvidos.

Como sabemos matematicamente as relações de fotocorrente determinadas por \autoref{eq_modEletFot} e \autoref{eq_modEletFotIl}, e tamb\'em as Figuras de M\'erito que caracterizam o fotodetector, podemos utilizar o APS para coletar as informações em sua sa\'ida e determinar a intensidade da luz

Entre os est\'agios 2 e 3 apresentados na \autoref{figura_estagiosAPS}, a tensão do n\'o $V_{cn}$ passa a diminuir, devido a circulação da fotocorrente. Como essa corrente depende da intensidade da luz, podemos descobrir o valor de intensidade sabendo a inclinação da curva de tensão na sa\'ida, j\'a que a variação da tensão \'e uma grandeza diretamente proporcional à corrente e a imped\^ancia vista no n\'o. Como a tensão de sa\'ida \'e igual a $V_{cn}$ menos $V_{GS}$ do transistor $T_{buffer}$, podemos medir a sa\'ida para processar o sinal e então relatar a intensidade luminosa.

Com as observações aqui apresentadas, podemos desenvolver um sistema de medição de informação luminosa, trabalhando entre os est\'agios 1 e 3, e então retornando ao Estágio 1 para uma nova aquisição. \'E importante destacar que existem limitações quanto \`a temporização dos est\'agios. Para que a medição seja realizada de forma adequada, devemos garantir que tenhamos entre o Est\'agio 1 e 2, um tempo suficientemente grande para que $V_{cn}$ apresente um valor estável, ou seja, o tempo de transição do sistema nessa condição seja conclu\'ido. Entre o Est\'agio 2 e Est\'agio 3, devemos garantir que tenhamos um tempo mínimo para que dois valores distintos de $V_{out}$ possam ser medidos, de acordo com a sensibilidade do sistema de medição.

\section{Amplificador de Transimped\^ancia (TIA)}
\label{section:TIA}

Um amplificador de transimped\^ancia \'e um circuito em que dada uma corrente de entrada, gera-se uma tensão em sua sa\'ida proporcional \`a esta corrente \cite{RazaviFundM}. Considerando-se o fotodiodo como uma fonte de corrente, a \autoref{fig_TIA} apresenta uma poss\'ivel topologia de um Amplificador de Transimped\^ancia (TIA), desenvolvido no presente trabalho.

\begin{figure}[!h]
	\caption{\label{fig_TIA}TIA desenvolvido}
	\begin{center}
	    \includegraphics[scale=0.3]{Circuitos/TIA.png}
	\end{center}
	\legend{Fonte da ilustração: Pr\'oprio autor}
\end{figure}

Onde:

\begin{itemize}
    \item $V_{ref}$ \'e a tensão de entrada de referência [$V$]
    \item $V_o$ \'e a tensão de sa\'ida [$V$]
    \item $R$ \'e uma resist\^encia de ajuste ganho [$\Omega$]
    \item $R_L$ \'e uma carga de sa\'ida [$\Omega$]
\end{itemize}

Observando a \autoref{fig_modelofotodiodo}, nota-se que podemos modelar uma resist\^encia $R_{sh}$ em paralelo ao fotodiodo. Desprezando-se todos outros componentes do modelo para facilitar a an\'alise, podemos descrever a f\'ormula correspondente ao $TIA$ como a \autoref{eqCTIA}.

\begin{equation}
    \label{eqCTIA}
    V_o = RI_{PH} + (1+\frac{R}{R_{sh}})V_{ref}
\end{equation}

Onde:
\begin{itemize}
    \item $I_{PH}$ \'e a corrente fotogerada [$A$]
\end{itemize}

A \autoref{eqCTIA} nos mostra que \textit{R} deve ser ajustado de forma que não seja grande demais, pois valores altos (de acordo com a aplicação) irão gerar tensão na sa\'ida com variação muito grande. Mesmo que Vref fosse colocado em GND, o fator $(1+\frac{R}{R_{sh}})$ ainda pode dominar a relação, devido \'a ruídos presentes no nó, e então impor um termo indesejado, o que também causaria problemas quando \textit{R} fosse muito alto \cite{hamamatsu}.

Devido ao produto $RI_{PH}$, também devemos ter o cuidado de não tornar \textit{R} pequeno demais para que tenhamos a sensibilidade no sinal de saída de forma adequada.

Considerando $R_{sh} >> R$, chegamos à \autoref{eqCTIA2}, utilizada no projeto apresentado no trabalho.

\begin{equation}
    \label{eqCTIA2}
    V_o = RI_{PH}
\end{equation}

\subsection{Resposta espectral de um TIA}

Como a imped\^ancia de entrada do $TIA$ não \'e ideal (infinita) e varia com a frequ\^encia e temperatura, o circuito não se apresenta linear em toda sua banda de operação, o que representa uma variação na resposta em frequ\^encia de acordo com a corrente fotogerada \cite{hamamatsu}.
Como o pr\'oprio amplificador apresenta capacit\^ancias internas, o circuito \'e tamb\'em limitado em altas frequ\^encias, principalmente pelo produto $R$ vezes $C$, onde C \'e a capacit\^ancia interna vista pelos terminais de $R$ \cite{hamamatsu}.

A \autoref{figura_respostaTIA} e a \autoref{figura_respostaTIA2} mostram o comportamento t\'ipico de resposta em frequ\^encia da topologia. O pico observado na \autoref{figura_respostaTIA} é característico de um sistema de 2ª ordem e acontece devido a capacitância vista nos terminais do fotodiodo, em conjunto com capacitâncias internas ao amplificador e na saída. Para reduzir esse efeito, um capacitor pode ser adicionado em paralelo ao resistor \textit{R}, com valor precisamente calculado de forma a atenuar o pico (formando um circuito denominado \textit{CTIA}). A\autoref{figura_respostaTIA2} mostra as oscilações presentes em resposta a um pulso, que da mesma forma, também podem ser atenuados utilizado o mesmo capacitor.

\begin{figure}[!h]
    \begin{minipage}{0.4\textwidth}
    \centering
    \caption{\label{figura_respostaTIA}Resposta espectral de um TIA}
	\includegraphics[scale=0.8]{Imagens/RespostaEspectralTIA.png}
	\legend{Fonte: \cite{hamamatsu}}
	\end{minipage}
	 \hfill
  \begin{minipage}{0.4\textwidth}
    \centering
    \caption{\label{figura_respostaTIA2}Resposta a um pulso em um TIA}
    \includegraphics[scale=0.8]{Imagens/RespostaEspectralTIA2.png}
    \legend{Fonte: \cite{hamamatsu}}
  \end{minipage}
  
	
\end{figure}