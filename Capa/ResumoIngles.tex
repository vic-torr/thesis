
% resumo em inglês
\begin{resumo}[Abstract]
 \begin{otherlanguage*}{english}

The Amazon Forest has suffered a worrying increase in the number of new cases of deforestation and artisanal mining on protected lands in the previous decade, in addition to fluvials on the Madeira River in 2022. As a result, monitoring agencies require modern techniques to identify irregularities capable of helping in the detection of small regions in the vast territory of the Amazon basin, for the maintenance of environmental preservation policies. New techniques of computer vision, pattern recognition and deep learning allow solutions of tasks at better levels of resolution and performance in the context of remote sensing.
  
Based on this field, this work studies the limitations of the already established techniques of convolutional networks and seeks to carry out a proof of concept of a computer vision technique of Swin architecture for classifying images of irregular areas of the Amazon, based on visual transformers. As well as the performance comparison between the two techniques. In addition to providing a methodology and reproducible experiments for using the Swin architecture for similar problems, from different sets of remote sensing data.

This work documented the experiment carried out with steps such as: aspects of creating the cloud environment using the GoogleColab platform in order to better exploit its computational resources; exploratory data analysis; build a base model by searching through different configurations, components and training features, using libraries such as Pytorch and SciKitLearn.

In this way, a base model of a ResNet50 convolutional neural network architecture with good performance was developed using classification metrics such as $F_2$, and PR curve, with the exception of the rare classes of the dataset, which had a poor performance. Following the same training settings, the second Swin architecture model was fitted and compared with the one based on ResNet50. It presented relevant performance improvements and better generalization with few samples under varied climate conditions in the classification of rare classes. Thus demonstrating the capability and compatibility of this architecture for the application.
  
\vspace{\onelineskip}

   \noindent 
   \textbf{Keywords}: Remote sensing; Visual Transformers; Transformer Swin; Deep learning; Convolutional Neural Networks; Machine Learning; Pattern Recognition; PyTorch; Computer vision.
 \end{otherlanguage*}
\end{resumo}