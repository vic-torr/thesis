
% resumo em inglês
\begin{resumo}[Abstract]
 \begin{otherlanguage*}{english}

The advent of technology offers many possibilities in different areas. The greatest benefit that was in automating tasks and improving activities that were previously done manually. In this sense, there has been a great increase in the use of robots to perform tasks in industrial, commercial and residential environments, due to lower costs compared to the last decades and the development of new technologies. Among these technologies, many seek to develop fields of Computer Vision, such as Object Detection. This work explores the implementation of Pattern Recognition combined with the control of Robotic Manipulators in order to always enable the best framing of a camera coupled to the end-effector. For this purpose, OpenCV and TensorFlow libraries are used to assist in capturing and identifying people in the image, in addition to the COCO-SSD pre-trained model that has a wide variety of classification of objects and a database updated from time to time. Furthermore, concepts of direct and differential kinematics are used in order to carry out the kinematic control of the manipulator. Validation comes through the simulation in CoppeliaSim platform, using programs such as Visual Studio Code and MATLAB to intermediate all data processing and manipulator control. The combination of these technologies expands the possibilities in industrial environments, in the use of security cameras and even in the entertainment industry, since it can guarantee a stabilization and tracking of pre-defined objects.

   \vspace{\onelineskip}
 
   \noindent 
   \textbf{Keywords}: Robotic Manipulator; Machine Learning; Pattern Recognition; TensorFlow; CoppeliaSim.
 \end{otherlanguage*}
\end{resumo}