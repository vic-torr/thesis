% ---
% RESUMOS
% ---

% resumo em português
% \setlength{\absparsep}{18pt} % ajusta o espaçamento dos parágrafos do resumo
\begin{resumo}
  

  A Floresta Amazônica sofreu um aumento preocupante número de novos casos de desmatamento e de garimpos em terras protegidas na década anterior, além de garimpos fluviais no rio madeira em 2022. Com isso, agências de monitoramento requerem modernas técnicas para a identificação de irregularidades capazes de auxiliar na detecção de pequenas regiões em no vasto território da bacia amazônica, para manutenção de políticas de preservação ambiental. Novas técnicas de visão computacional, reconhecimento de padrões e aprendizado profundo permitem soluções e a realização de tarefas em  melhor nível de resolução e desempenho no contexto de sensoreamento remoto.
  
  Partindo desse ponto, este trabalho estuda as limitações das técnicas já estabelecidas de redes convolucionais e realizará uma prova de conceito de uma técnica de visão computacional de arquitetura Swin para classificação de imagens de áreas irregulares da amazônia, baseado em transformers visuais. Assim como a comparação de desempenho dentre as duas técnicas. Além de prover uma metodologia e experimentos reprodutíveis para utilização da arquitetura Swin para problemas análogos, de diferentes conjuntos de dados de sensoriamento remoto.
    
  Este trabalho documentou o experimento realizado com etapas como: aspectos da criação do ambiente em nuvem utilizando a plataforma GoogleColab para melhor explorar seus recursos computacionais; análise exploratória dos dados; construção um modelo base vasculhando diferentes configurações, componentes e características de treinamento, utilizando bibliotecas como Pytorch e SciKitLearn.
  
  Dessa forma, foi desenvolvido um modelo base de uma arquitetura de rede neural convolucional ResNet50 com bom desempenho global utilizando métricas de classificação como a $F_2$, e curva PR. Contudo, as classes raras do conjunto de dados, tiveram um mal desempenho. Seguindo as mesmas configurações de treino, o segundo modelo de arquitetura Swin foi ajustado e comparado com o baseado em ResNet50. Apresentou relevantes melhorias de desempenho e melhor capacidade de generalização com poucas amostras sob condições climáticas variadas na classificação de classes raras. Demonstrando assim a  compatibilidade desta arquitetura para a aplicação.

  
\vspace{\onelineskip}

\noindent 

\textbf{Palavras-chave}: Sensoriamento remoto; Transformers Visuais; Transformer Swin; Aprendizado profundo; Redes Neurais Convolucionais; Aprendizado de Máquina; Reconhecimento de Padrões; PyTorch; Visão computacional.
\end{resumo}