% ---
% RESUMOS
% ---

% resumo em português
% \setlength{\absparsep}{18pt} % ajusta o espaçamento dos parágrafos do resumo
\begin{resumo}

O advento da tecnologia proporcionou muitas possibilidades em diversas áreas. O maior aproveito que se teve foi na automatização de tarefas e melhoria de atividades que antes eram feitas manualmente. Nesse sentido, houve um grande aumento do uso de robôs para realizar tarefas em ambientes industriais, comerciais e residenciais, devido ao barateamento de custos e desenvolvimento de novas tecnologias. Dentre essas tecnologias, muitas buscam desenvolver campos da Visão Computacional, como a Detecção de Objetos. Este trabalho explora a implementação de Reconhecimento de Padrões aliado ao controle de Manipuladores Robóticos a fim de possibilitar sempre o melhor enquadramento de uma câmera acoplada no efetuador. Para isso, são utilizadas as bibliotecas OpenCV e TensorFlow para auxiliar na captura e identificação de pessoas na imagem, além do modelo pré-treinado COCO-SSD que possui vasta variedade de classificação de objetos e um banco de dados atualizado de tempos em tempos. Além disso, são utilizados conceitos de cinemática direta e diferencial a fim de se realizar o controle cinemático do manipulador. A validação vem por meio da simulação através da plataforma CoppeliaSim, utilizando programas como Visual Studio Code e MATLAB para intermediar todo o processamento de dados e controle do manipulador. A junção dessas tecnologias amplia as possibilidades em ambientes industriais, no uso de câmeras de segurança e até mesmo na indústria do entretenimento, uma vez que se pode garantir uma estabilização e rastreamento de objetos pré-definidos.

\vspace{\onelineskip}
 
  \noindent 

 \textbf{Palavras-chave}: Manipulador Robótico; Aprendizado de Máquina; Reconhecimento de Padrões; TensorFlow; CoppeliaSim.
\end{resumo}