\section{Circuitos de Teste}
\label{BlocoTestes}

O circuito apresenta dois circuitos de teste adicionais, sendo um para um APS e outro para o TIA. A finalidade destes circuitos \'e de testar o sistema sem a necessidade de uma fonte luminosa. Para isso \'e acrescentado um pino diretamente aos catodos dos fotodiodos, em que uma tensão pode ser diretamente injetada nos mesmos de forma a simular a fotogeração.

O APS de teste utiliza os pinos de RESET e ENABLE iguais ao do APS de cor azul, mas \'e polarizado com uma fonte de corrente pr\'opria e também apresenda sa\'idas pr\'oprias. Caso se mantenha o pino de injeção de corrente flutuante, o APS deve funcionar de forma equivalente aos outros.