\renewcommand{\NomeBloco}{\textit{APS}}
\renewcommand{\NomeBlocoNoUnderline}{blocoAPS}
\renewcommand{\NomePTab}{tab_\NomeBlocoNoUnderline}
\renewcommand{\NomeSTab}{tab_\NomeBlocoNoUnderline2}
\renewcommand{\NomePFig}{fig_\NomeBlocoNoUnderline}
\renewcommand{\NomeSFig}{fig_\NomeBlocoNoUnderline2}
\renewcommand{\NomeTTab}{tab_\NomeBlocoNoUnderline3}
\renewcommand{\NomeQTab}{tab_\NomeBlocoNoUnderline4}

\section{APS}

O bloco \NomeBloco{} implementa o circuito apresentado na \autoref{fig_APS}. O bloco apresenta as definições de sinais de entrada e sa\'ida referidos na \autoref{\NomeSTab}.

\begin{table}[!h]
\label{\NomeSTab}
\begin{tabular}{ccll}
\toprule
Sinal      & Tipo    & \multicolumn{1}{c}{Descrição}                                                          & \multicolumn{1}{c}{Observação}                                                                               \\
\midrule \midrule
RESET      & Entrada & Sinal de tensão de RESET no APS                                                        & Ativo em nível baixo                                                                                         \\
\midrule
ENABLE     & Entrada & Sinal de tensão de ENABLE no APS                                                       & Ativo em nível alto                                                                                          \\
\midrule
SELECT     & Entrada & Sinal de tensão de SELECT no APS                                                       & \begin{tabular}[c]{@{}l@{}}Ativo em nível alto.\\ Mantido internamente \\ em nível alto\end{tabular} \\
\midrule
Ibias\_clk & Entrada & Dreno de Corrente do bloco (500 nA)                                                    &                                                                                                              \\
\midrule
Vout       & Saída   & \begin{tabular}[c]{@{}l@{}}Sinal de tensão analógica produzido\\ pelo APS\end{tabular} &    \\
\bottomrule
\end{tabular}
\legend{Fonte: Produzido pelo autor.}
\end{table}

A representação em bloco do circuito projetado é dado na \autoref{\NomeSFig}.

\begin{figure}[!h]
 \centering
    \centering
    \caption{\label{\NomeSFig}Representação em bloco do \NomeBloco} 
    \includegraphics[scale=0.3]{Circuitos/APS_block.png}
    \legend{Fonte: Produzido pelo autor}
\end{figure}

Os transistores utilizados no bloco apresentam os par\^ametros mostrados na \autoref{\NomeTTab}.

\begin{table}[!h]
\caption{Transistores do bloco \NomeBloco}
\label{\NomeTTab}
\centering
\begin{tabular}{ccccc}
\toprule
Transistor & W ($\mu$m)  & L ($\mu$m)           & M (n° dispositivos) & S (n° dispositivos)\\
\midrule \midrule
T\textsubscript{buffer} & 15 & 0,75 & 1 & 1\\
\midrule
T\textsubscript{reset} & 4 & 0,18 & 1 & 1\\
\midrule
T\textsubscript{select} & 10 & 0,18 & 2 & 2\\
\bottomrule
\end{tabular}
\legend{Fonte: Produzido pelo autor}
\end{table}

O fotodiodo utilizado no bloco apresenta os par\^ametros mostrados na \autoref{\NomeQTab}. $W$ é a largura do fotodiodo. $L$ é o comprimento do fotodiodo.

\begin{table}[!h]
\caption{Fotodiodo do bloco \NomeBloco}
\label{\NomeQTab}
\centering
\begin{tabular}{cccc}
\toprule
Nome & W ($\mu$m)  & L ($\mu$m) & Área ($\mu$m²)\\
\midrule \midrule
Fotodiodo & 25 & 25 & 625\\
\bottomrule
\end{tabular}
\legend{Fonte: Produzido pelo autor}
\end{table}

Informações referentes à $T\textsubscript{enable}$ podem ser vistas no \autoref{portatg}.