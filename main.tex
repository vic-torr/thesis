\include{Configuracoes/ConfiguracoesABNT}

% ---
% Informações de dados para CAPA e FOLHA DE ROSTO
% ---
\titulo{Implementação de um modelo de classificação de áreas irregulares na floresta amazônica}
\autor{Victor José Ferreira de Moraes}
\local{Belo Horizonte, Minas Gerais, Brasil}
\data{25 de Maio de 2022}
\orientador{Cristiano Castro}
\coorientador{}
\instituicao{%
  Universidade Federal de Minas Gerais \- UFMG
  \par
  Escola de Engenharia
  \par
  Departamento de Engenharia Elétrica
 }
\tipotrabalho{Monografia (Graduação)}
% O preambulo deve conter o tipo do trabalho, o objetivo, 
% o nome da instituição e a área de concentração 
\preambulo{Monografia de conclusão de curso para obtenção do nível de Bacharel em Engenharia Elétrica pela Escola
de Engenharia da Universidade
Federal de Minas Gerais.}
% ---

\include{Configuracoes/ConfiguracoesAparencia}

% ----
% Início do documento
% ----
\begin{document}
%%%% ! %TEX root = main.tex

% Seleciona o idioma do documento (conforme pacotes do babel)
%\selectlanguage{english}
\selectlanguage{brazil}

% Retira espaço extra obsoleto entre as frases.
\frenchspacing

% ----------------------------------------------------------
% ELEMENTOS PRÉ-TEXTUAIS
% ----------------------------------------------------------
% \pretextual

% ---
% Capa
% ---
\imprimircapa
% ---

% ---
% Folha de rosto
% (o * indica que haverá a ficha bibliográfica)
% ---
%\imprimirfolhaderosto*
% ---

\input{Capa/FichaCatalografica}

\input{Capa/FolhaAprovacao}

% ---
% Dedicatória
% ---
\begin{dedicatoria}
   \vspace*{\fill}
   \centering
   \noindent
   \textit{Este trabalho é dedicado a todas as pessoas que,\\
por ventura, cruzaram meu caminho e acreditaram\\
que eu poderia ir muito além do que\\
eu imaginava.} \vspace*{\fill}
\end{dedicatoria}
% ---


% ---
% Agradecimentos
% ---
\begin{agradecimentos}
% Gostaria de agradecer a Deus e todas as pessoas que contribuíram, de forma direta ou indireta, para que este trabalho e toda esta trajetória fosse possível.

% Minha família, que me ensinou a base que pude levar por toda minha vida e me acompanhou por todo meu crescimento como pessoa e como profissional.

% Meus amigos, que compreende uma família a qual não compartilho o mesmo sangue. Estiveram sempre presentes nos piores e nos melhores momentos, incentivando a seguir em frente e me aconselhando a parar quando necessário.

% A todos os profissionais envolvidos na universidade, sejam estes professores, porteiros, faxineiros, entre outros. Cada um zelou por oferecer o melhor caminho possível para que, junto aos alunos, a UFMG se tornasse a melhor universidade federal do país.

% E finalmente, obrigado a sociedade que confia nos alunos de universidades públicas para se tornarem profissionais de qualidade, e retribuírem para sua população todo o conhecimento adquirido em suas trajetórias, a fim de vivenciarmos uma sociedade melhor.

% Obrigado a todos que acreditaram em nós.

\end{agradecimentos}
% ---

% ---
% Epígrafe
% ---
\begin{epigrafe}
    \vspace*{\fill}
	\begin{flushright}
		\textit{Qualquer tecnologia suficientemente avançada \\
		é indistinguível de magia.}
        
        Arthur C. Clarke
	\end{flushright}
\end{epigrafe}
% ---

% ---
% RESUMOS
% ---

% resumo em português
% \setlength{\absparsep}{18pt} % ajusta o espaçamento dos parágrafos do resumo
\begin{resumo}
  

  A Floresta Amazônica sofreu um aumento preocupante número de novos casos de desmatamento e de garimpos em terras protegidas na década anterior, além de garimpos fluviais no rio madeira em 2022. Com isso, agências de monitoramento requerem modernas técnicas para a identificação de irregularidades capazes de auxiliar na detecção de pequenas regiões em no vasto território da bacia amazônica, para manutenção de políticas de preservação ambiental. Novas técnicas de visão computacional, reconhecimento de padrões e aprendizado profundo permitem soluções e a realização de tarefas em  melhor nível de resolução e desempenho no contexto de sensoreamento remoto.
  
  Partindo desse ponto, este trabalho estuda as limitações das técnicas já estabelecidas de redes convolucionais e realizará uma prova de conceito de uma técnica de visão computacional de arquitetura Swin para classificação de imagens de áreas irregulares da amazônia, baseado em transformers visuais. Assim como a comparação de desempenho dentre as duas técnicas. Além de prover uma metodologia e experimentos reprodutíveis para utilização da arquitetura Swin para problemas análogos, de diferentes conjuntos de dados de sensoriamento remoto.
    
  Este trabalho documentou o experimento realizado com etapas como: aspectos da criação do ambiente em nuvem utilizando a plataforma GoogleColab para melhor explorar seus recursos computacionais; análise exploratória dos dados; construção um modelo base vasculhando diferentes configurações, componentes e características de treinamento, utilizando bibliotecas como Pytorch e SciKitLearn.
  
  Dessa forma, foi desenvolvido um modelo base de uma arquitetura de rede neural convolucional ResNet50 com bom desempenho global utilizando métricas de classificação como a $F_2$, e curva PR. Contudo, as classes raras do conjunto de dados, tiveram um mal desempenho. Seguindo as mesmas configurações de treino, o segundo modelo de arquitetura Swin foi ajustado e comparado com o baseado em ResNet50. Apresentou relevantes melhorias de desempenho e melhor capacidade de generalização com poucas amostras sob condições climáticas variadas na classificação de classes raras. Demonstrando assim a  compatibilidade desta arquitetura para a aplicação.

  
\vspace{\onelineskip}

\noindent 

\textbf{Palavras-chave}: Sensoriamento remoto; Transformers Visuais; Transformer Swin; Aprendizado profundo; Redes Neurais Convolucionais; Aprendizado de Máquina; Reconhecimento de Padrões; PyTorch; Visão computacional.
\end{resumo}


% resumo em inglês
\begin{resumo}[Abstract]
 \begin{otherlanguage*}{english}

The Amazon Forest has suffered a worrying increase in the number of new cases of deforestation and artisanal mining on protected lands in the previous decade, in addition to fluvials on the Madeira River in 2022. As a result, monitoring agencies require modern techniques to identify irregularities capable of helping in the detection of small regions in the vast territory of the Amazon basin, for the maintenance of environmental preservation policies. New techniques of computer vision, pattern recognition and deep learning allow solutions of tasks at better levels of resolution and performance in the context of remote sensing.
  
Based on this field, this work studies the limitations of the already established techniques of convolutional networks and seeks to carry out a proof of concept of a computer vision technique of Swin architecture for classifying images of irregular areas of the Amazon, based on visual transformers. As well as the performance comparison between the two techniques. In addition to providing a methodology and reproducible experiments for using the Swin architecture for similar problems, from different sets of remote sensing data.

This work documented the experiment carried out with steps such as: aspects of creating the cloud environment using the GoogleColab platform in order to better exploit its computational resources; exploratory data analysis; build a base model by searching through different configurations, components and training features, using libraries such as Pytorch and SciKitLearn.

In this way, a base model of a ResNet50 convolutional neural network architecture with good performance was developed using classification metrics such as $F_2$, and PR curve, with the exception of the rare classes of the dataset, which had a poor performance. Following the same training settings, the second Swin architecture model was fitted and compared with the one based on ResNet50. It presented relevant performance improvements and better generalization with few samples under varied climate conditions in the classification of rare classes. Thus demonstrating the capability and compatibility of this architecture for the application.
  
\vspace{\onelineskip}

   \noindent 
   \textbf{Keywords}: Remote sensing; Visual Transformers; Transformer Swin; Deep learning; Convolutional Neural Networks; Machine Learning; Pattern Recognition; PyTorch; Computer vision.
 \end{otherlanguage*}
\end{resumo}

% ---

% ---
% inserir lista de ilustrações
% ---
% \pdfbookmark[0]{\listfigurename}{lof}
%\listoffigures*
% \cleardoublepage
% ---

% ---
% inserir lista de tabelas
% ---
% \pdfbookmark[0]{\listtablename}{lot}
% \listoftables*
% \cleardoublepage
% ---

% ---
% inserir lista de abreviaturas e siglas
% ---
\begin{siglas}
  
  \item[VANT] Veiculo aéreo não tripulado
  \item[UAV] do inglês \textit{Unmanned Aerial Vehicle}
  \item[AVL] Localização Absoluta Visual do inglês \textit{Absolute Visual Location}
  \item[MINDS\textsuperscript{Lab}] Laboratório de Machine learning, Inteligencia computacional e Data Science
  \item[UFMG] Universidade Federal de Minas Gerais
 
\end{siglas}
% ---

% ---
% inserir lista de símbolos
% ---
\begin{simbolos}
  \item[$ P $] Classes positivas
    \item[$ P $] Classes negativas
      \item[$ TP $] classificações verdadeiro positivas
      \item[$ TN $] classificações verdadeiro Negativas
      \item[$ FP $] classificações falso positivas
      \item[$ FN $] classificações falso Negativas
    \item[$ F_2 $] Métrica F2 de desempenho baseadas em índices de precisão e revocação. 
  
\end{simbolos}
% ---

% ---
% inserir o sumario
% ---
%\pdfbookmark[0]{\contentsname}{toc} - Não remover comentário
% \tableofcontents*
% \cleardoublepage
% ---

% ----------------------------------------------------------
% ELEMENTOS TEXTUAIS
% ----------------------------------------------------------
\textual
\chapter{Introdução}
\label{sec:introducao}

VANT ou veiculos autonomos aéreos não tripulados possuem crescente potencial e  aplicação nas indústrias agricolas,
militares e serviços. E para que sua navegação seja realizável, é imprescindível a implementação de um sistema de
localização robusto [citar business report]. As formas de localização mais amplamente utilizadas são a partir de GPS
podendo ser combinados ou não com fusão sensorial à uma IMU \-- Unidade de medição inercial. O GPS embora tenha
acurácia da ordem de poucos metros, possui o compromisso de ter baixa taixa de amostragem. Isso pode ser mitigado com a
fusão sensorial com medições da IMU, assim obtendo melhores estimativas de posição. A estimação de posição utilizando somente a IMU também pode não ser adequada pelo fato da estimativa ser calculada integrando as medições de aceleração e assim possuindo alto drift. Dessa forma, essa estimação clássica de posição é mais adequada pelo conjunto de ambos.
Este conjunto, contudo, deixa de ser preciso quando não há a disponibilidade de GPS. Essas condição pode ocorrer por
eventuais indisponibilidades, também por interferencias destrutivasm comuns em um ambiente urbando ou por signal jamming
deliberadamente. Ambientes urbanos possuem superfícies metálicas que refletem o sinal transmitido pelos satélite. Tal
sinal refletido ao chegar no receptor GPS pode gerar interferências destrutivas com o sinal original desejado. Signal jamming é uma vulnerabilidade comum para drones, seja de aplicações civis ou militares, já que um agente maliciosoo pode desabilitar a localização ou comunicação do drone por meio de um sinal de interferência. 


Para que a localização seja realizada, é necessário que o drone possua um sistema de localização robusto. E neste
trabalho será proposto e implementado um sistema de localização via pattern matching complementar as localizações
anteriormente citadas. O patern matching será feito a partir da comparação das imagens de georeferendiadas de satélite com as imagens captturadas instantaneamente. Além disso, terá como escopo portar o modelo de reconhecimento de padrões para o hardware limitado do drone e metodologias de reduzir o custo computacional de avaliação de padrões.


Todo trabalho será desenvolvido com apoio do laboratório MINDS (Machine Intelligence and Data Science) da Universidade Federal de Minas Gerais interno a Escola de Engenharia, que possui o ferramental e orientação.

A necessidade de atribuir a dispositivos tarefas que fazemos no dia a dia se tornou algo comum na atualidade. Isso se deve a uma evolução que se mostrou evidente tanto em termos de hardware quanto de software, aparecendo produtos no mercado que atendem tanto a demanda industrial quanto a comercial. E, na medida em que certa tecnologia cresce e tem o aumento da comunidade desenvolvedora trabalhando em conjunto, ela acaba se tornando cada vez mais acessível, possibilitando uma maior interação interdisciplinar.

Dentro deste contexto, pode-se observar a crescente aplicação de manipuladores robóticos em setores diversificados. O uso mais evidente é dentro do contexto industrial, uma vez que os manipuladores robóticos podem automatizar processos antes feitos manualmente pelos funcionários, podendo assim oferecer agilidade, segurança, dentre outros benefícios às empresas. Porém, pode-se observar também que estes estão sendo utilizados cada vez mais comercialmente, como é o caso de \textit{gimbals}, \textit{hoverboards} e \textit{drones}.

Ainda na vertente do crescimento tecnológico, os equipamentos atualmente possuem hardware potentes a preços acessíveis, capazes de rodar aplicações complexas que vão de encontro a popularização de ideias, como a Internet das Coisas. Os dispositivos permitem cada vez mais essa integração, tanto com aplicações industriais quanto domésticas. Essa popularização permitiu que várias bibliotecas fossem criadas e publicadas para a comunidade desenvolvedora a fim de instigar mais o desenvolvimento de possiblidades criativas para o mercado.

Graças a performance dos dispositivos atuais, houve um aumento de aplicações que exigem uma maior complexidade no processamento de dados. Um exemplo que vem crescendo nos últimos anos é o uso de reconhecimento de padrões, a qual vem sendo utilizado em meios de produção e até mesmo como meio de entretenimento em redes sociais com o uso de filtros digitais. A finalidade do presente trabalho é desenvolver um software que utilize de ferramentas de reconhecimento de padrões a fim de entregar parâmetros para controlar um manipulador robótico.

Como a maior parte da tecnologia é lançada com o objetivo de entregar mais conforto aos usuários, muitas fabricantes têm apostado em dispositivos que utilizam da tecnologia \textit{hands-free}. Ela possibilita que os usuários desfrutem de seus produtos sem a necessidade de tocá-los, sendo a maior aposta nos comandos de voz.

Com o maior acesso a tecnologia, houve um crescente uso de aplicações com câmera. Isso foi acentuado durante a pandemia de COVID-19, elevando o número de chamadas de vídeo realizadas ao redor do mundo devido às restrições sanitárias. Com intuito de integrar a tecnologia \textit{hands-free} na utilização de câmeras em dispositivos móveis, o software desenvolvido neste projeto tem o objetivo de enquadrar as pessoas na tela, permitindo que se movimentem e realizem outras tarefas durante a captura de imagem. Para isso, o dispositivo será instalado no manipulador robótico que, a partir do uso de reconhecimento facial, recebe comandos de controle para posicionar a câmera de forma adequada.

Este trabalho abrange quatro capítulos além deste. O segundo capítulo contém a revisão bibliográfica para o estudo de caso proposto. O terceiro capítulo aborda a metodologia adotada para a realização da proposta e o desenvolvimento do software. Adiante, o quarto capítulo, contém a análise dos resultados esperados. Por fim, o capítulo cinco conclui este trabalho e aponta direções futuras para esta área de pesquisa.




\chapter{Revisão Bibliográfica}
\label{sec:ref_teorico}
Neste capítulo, será apresentado inicialmente definições do domínio do problema de classificação e identificação de cenas em sensoriamento remoto na seção~\ref{sec:Cap2_dominio}, bem como suas características e contexto. Seguindo por um apanhado teórico da seção envolvendo aprendizado de máquina, redes neurais, redes neurais profundas, convolucionais e \textit{transformers}, no que tange interseções com possiveis soluções para o problema. E por fim, na seção~\ref{sec:Cap2_revisao_literatura} uma revisão da literatura e do atual estado da arte no que tange o problema de classificação visual no que tange a sensoriamento remoto. Apresentando trabalhos que envolveram soluções tanto para sensoriamento remoto, quanto para implementações de redes neurais profundas.


% trabalhos:
% - transformer pra label da amazonia
% - unet pra detecção de garimpo,
% - attention is all you need,
% - a image worth 16x16,
% - paper principal da cnn alexnet,
% - solução dessse problema via cnn.

\section{\textit{Domínio do problema}}\label{sec:Cap2_dominio}


% TODO: fonte definição sensoriamento remoto



O problema em questão, de identificação de regiões com garimpo via imagens de satélite, é contido no campo de sensoriamento remoto. O sensoriamento remoto consiste em aquisitar ou analisar medições de uma região geografica terrestre ou atmosférica. Podem ser realizadas por imagens aéreas ou por satélites, podendo abranger diferentes partes do espectro eletromagnético. Frequentemente consistem em métodos de aquisição ou processamento de sinais e imagens, para obter caracteristicas ou reconhecer padrões em tais localidades~\cite{emery2017introduction}.O termo foi cunhado fazendo referencia a medições realizada por algum meio indireto ou “remoto”, em vez de de um contato direto com sensores no ambiente medido~\cite{emery2017introduction}.

Ainda mais especificamente, o problema indroduzido também faz parte do campo de reconhecimento de padrões em imagens. Temos que a área varrida por sensoriamento remoto é extensiva para ser vasculhada manualmente. Portanto se faz necessário o uso de algorítimos que automatizem a detecção ou classificação do objeto a ser encontrado.
O campo de reconhecimento de padrões possui algorítimos tradicionais para identificar e localizar objetos de interesse, porém podem ser muito caras computacionalmente e pouco efetivas, dependendo das características do problema. Por isso, recentemente tem sido frequentemente empregadas técnicas baseadas em aprendizado de máquina e aprendizado profundo, que atualmente são o estado da arte em muitos problemas de reconhecimento de padrões.


\section{Revisão Teórica}\label{sec:Cap2_revisao_teorica}

\subsection{Aprendizado de máquina}\label{sec:aprendizado_maquina}

Aprendizado de máquina é um campo que estuda algorítimos capazes de aprender e realizar inferências a partir de dados. O termo foi cunhado em 1959, por Artur Samuel, como “Campo de estudos que visa a dar computadores a habilidade de aprender sem serem explicitamente programados para determinada tarefa.”~\cite{Samuel1959SomeSI}.

Já em~\cite{Mitchell97} definie um algorítimo de aprendizdo como “Um algorítmo que é dito ser capaz de uma experiência E com respeito a determinada classe de tarefas T e com medidas de performance P, se sua performance nas tarefas em T, medidas por P, melhoram a partir da experiência E.”

Para melhor entender cada constituinte dessa definição, podemos enumerar exemplos. A
tarefa T temos como exemplo o problema de classificação, que consiste do algorítimo
responder quais das k categorias para o qual ele experimentou em E, certas amostras de entradas pertencem. Para resolver tal tarefa, tal algorítimo de aprendizado deve produzir uma função $f:\Re^n\rightarrow \{1,\ldots,k\}$. Quando $y=f(x)$, o modelo atribui uma entrada descrita pelo por $x$, que nosso caso constitui uma amostra de entrada, a uma categoria identificada por um código numerico $y$. Uma variante do mesmo problema é em vez de classificar qual classe, retornar a distribuição de probabilidade sobre as classes.\cite{GoodBengCour16}. A medida de performance P, necessária para avaliar as habilidades de um dado algorítimo. Tal medida de performance geralmente é atrelada ao tipo de tarefa sendo realizada pelo sistema. Para tarefas de clssificação, é comumente utilizada a métrica de acurácia do modelo. Consiste na proporção de amostras na qual o modelo produz a saída correta. Outras métricas também podem ser interessantes em casos onde amostras das classes são muito desbalanceadas, como escore f1 e curva precisão-revocação.


\subsection{Redes Neurais Artificiais}\label{sec:Cap2_redes_neurais}

O termo redes neurais embarca uma grande classe de modelos e métodos de aprendizados. O modelo mais simples, também podendo ser chamado rede de camada oculta única de perceptrons. Já o  perceptron é a célula de uma rede neural. Se trata de um modelo matemático análogo a um neurônio. Possui um vetor de entradas e sobre essas entradas é aplicada uma combinação linear utilizando os pesos sobre cada entrada. O resultado de tal operação por sua vez passa por uma função de ativação que resulta numa saída binária de classificação do perceptron. Dessa forma, ao se otimizar os pesos e função de ativação a determinadas amostras de treino e suas respectivas saídas rotuladas, podemos criar um classificador linear simples, caso seja um problema linearmente separável. Portanto o processo de treinamento de uma rede neural se trata de otimizar os pesos dos neurônios. Uma etapa importante dos ajustes dos pesos é a retro-propagação de erro\footnote{Backpropagation} que é um algoritmo que otimiza os pesos da camada oculta~\cite{hastie01statisticallearning}.

\subsection{Redes Neurais Artificiais Profundas}\label{sec:Cap2_redes_neurais_profundas}
O termo redes neurais profundas e o aprendizado profundo, ou comumente chamado de \textit{deep learning}, se refere a redes neurais artificiais com múltiplas camadas ocultas. Foram uma das tecnologias maior desenvolvidas nos últimos anos, e se tornaram cada vez mais popular. Devido a sua superior \textit{performance} em extração de características, teve sucesso por distintos domínios, como visão computacional, reconhecimento de fala, processamento natural de linguagem e em big data.

Um dos riscos envolvendo o treinamento de redes neurais profundas é o problema de \textit{overfiting}\footnote{Sobre-ajuste}. Se trata de quando o modelo é treinado e de forma a gerar uma função próxima demais aos dados de treino, e perdem generalidade, falhando em predições em dados fora do conjunto de treino. Para mitigar o surgimento de \textit{overfiting} durante o treino, são utilizadas técnicas de regularização. Consistem em adicionar penalidade à complexidade do modelo, de forma que o treino otimize para se tornar uma função genérica. Dentre as técnicas de regularização possíveis de DNNs, podemos citar o Dropout, Dropconnect e pruning, que são respectivamente a remoção, adição de conexão entre neurônios e remoção de neurônios~\cite{hastie01statisticallearning}.


\subsection{Redes Neurais Convolucionais}\label{sec:Cap2_redes_neurais_convolucionais}
Uma arquitetura clássica é a rede neural convolucional (CNN), que utiliza convoluções para extrair características de uma imagem entre cada camada de filtros. Também possui camadas de \textit{pooling}, não lineares e camadas completamente conectadas~\cite{8308186}. Uma dos pressupostos das \textit{CNNs} é os filtros serem indiferentes a translações das características na imagem, possibilitando assim uma eficiente extração de características para composição e identificação da imagem.


\subsection{\textit{ResNet}}\label{sec:Cap2_ResNet}
% TODO Resnet

\subsection{\textit{O problema de rotulagem e variabilidade de amostras de treino}}\label{sec:Cap2_rotulagem}

Um dos principais desafios envolvido o treino de \textit{CNNs} aplicado a sensoriamento remoto é representar um estado de características que cubram as variações fotográficas, tanto em características do sensor, como variações da imagem no dia, clima, estação e plataforma da câmera o que se torna uma tarefa difícil. Para uma localização efetiva, o modelo deve ser robusto a todas essas variações, que requer um grande conjunto de treino que cobre boa parte das diversas condições possíveis. Tal conjunto de dados não é disponível e nem viável de obter, pois se trata um volume muito grande de amostras~\cite{rs13194017}. Tais limitações levam a necessitar o desenvolvimento de algorítimos que aprendem seletivamente para que o poder computacional seja utilizado eficientemente, bem como reutilizar conhecimento prévio e evitar treinamento redundante~\cite{rostami2019learning}.  Dentre as técnicas utilizadas para implementar esses modelos mais eficientes, temos como exemplo o aprendizado supervisionado fraco, e a transferência de aprendizado.

\subsection{\textit{Aprendizado semi-supervisionado}}\label{sec:Cap2_semisup}

As técnicas de aprendizado semi-supervisionado consistem em treinar um modelo com apenas um conjunto reduzido de amostras rotuladas de treino, e as demais amostras serem não supervisionadas. As demais amostras de treino podem ser utilizadas, por exemplo, agrupando-as com as amostras rotuladas e classificando-as como a amostra mais próxima, como apresenta o trabalho de~\cite{Sanches2003}. Outras propostas envolvem data augmentation, que consistem em gerar um conjunto de treino maior, dados as amostras de treino disponíveis.

\subsection{\textit{Transferência de aprendizado}}\label{sec:Cap2_transfer}

Já técnicas de transferência de aprendizado\footnote{Comumente citado como Transfer Learning}, ou \textit{few shots learning}, consistem em redes treinadas para um conjunto limitado de testes~\cite{rostami2019learning}
e reutilizam esse conhecimento através de diferentes domínios, tarefas ou agentes. Consistem primariamente de um problema origem e um problema objetivo e como podemos suceder em transferir conhecimento dado o problema origem. A abordagem de transferência de conhecimento se inspira em replicar a habilidade dos humanos em que é possível transferir conhecimento de experiências passadas para lidar com tarefas com poucas amostras rotuladas. Este fato inspirou em representar dados de diferentes problemas de aprendizado de máquina em um espaço embutido onde as representações utilizam de diversas relações entre diferentes domínios de conhecimentos e tarefas. Uma implementação encontrada na literatura, proposta em~\cite{DBLP:journals/corr/abs-1811-04863}, consistiu de transplantar a camada de características de uma CNN derivada do domínio de origem para inicializar outra rede, do domínio objetivo, composta por uma camada final fortemente conectada. Assim foi aproveitada as primeiras camadas e a rede foi trenada para o domínio objetivo com uma quantidade menor de amostras.

\section{Revisão da Literatura}\label{sec:Cap2_revisao_literatura}

\subsection{\textit{Problema de detecção em sensoriamento remoto em outros trabalhos}}\label{sec:Cap2_outros_trabalhos}


O problema de classificação para detecção em sensoriamento remoto, utilizando aprendizado profundo~\cite{s20236936}


% TODO: citar o paper de pre-treino
Para a aplicação de sensoriamento remoto temos trabalhos~\cite{wang2022empirical} que demonstram a aplicação de transferência de aprendizado e redes pré-treinadas. Utilizou-se o dataset MillionAID, que é o maior dataset datado até agora, contendo mais de um milhão de imagens sem sobreposição, com múltiplas visões temporais para a mesma cena, de canais apenas RGB. Possui uma árvore de classificação com 51 folhas, de cenas de terras de: agricultura, comercial, industrial, serviço público, industrial, transporte, regiões com água e regiões inutilizadas. Cada Folha possui 2.000~45,000 imagens. Foram obtidas pelo Google earth, que possui uma diversidade de sensores, resultando em diferente resoluções, desde 50cm à 150m. E tamanhos de imagem de 10k pixels até 900 mega pixels.




\subsection{\textit{PyTorch}}\label{sec:Cap2_PyTorch}
Pytorch é um framework código aberto de Python para aprendizado de maquinas. É implementado em C++ e CUDA para otimizações de computações numéricas e matriciais, que são extensivas para esta aplicação.
Foi desenvolvido pelo Facebook e é atualmente amplamente adotado pela comunidade de pesquisa e mercado de aprendizado de máquina. Possui muitas implementações das ferramentas mais utilizadas, especificamente aplicadas a deeplearning e visão computacional. Tem ampla adoção devido a intenção de ser um framework de fácil uso e alto nível, com muitas abstrações e técnicas já implementadas.



% INPE


% - DETER B INPE Este documento apresenta a metodologia para o sistema DETER- B como parte
% do plano de desenvolvimento do Sistema DETER (Detecção de Desmatamento em
% Tempo Quase Real). Atualmente, o DETER utiliza imagens do sensor MODIS/TERRA,
% com alta frequência temporal, mas limitada resolução espacial (250 m), para
% mapeamento diário das áreas desflorestadas em formações florestais na Amazônia.
% Para organizar o processo de aprimoramento dos sistemas de alerta do INPE o DETER
% baseado no MODIS, que tem como área mínima de mapeamento de 0,25 km2 (25 ha)
% passa a ser denominado DETER-A. A exatidão dos alertas do Sistema DETER-A é maior
% que 90\%, sendo que aproximadamente 65% correspondem a desmatamento por corte
% raso e 30\% a evidências de degradação florestal (INPE, 2008).
% Deste modo, dentro da família de sistemas de alerta, pretende-se desenvolver e
% operacionalizar dois novos sistemas, o DETER-B que utilizará sensores com resolução
% de 60m e o DETER-C com dados de sensores da classe LANDSAT, de 20 a 30 m.


% plataforma mapbiomas




% ---
% Segundo o MapBiomas (https://plataforma.brasil.mapbiomas.org/):
% -
% A Amazônia concentra 94% (mais de 100 mil hectares) da área
% garimpada brasileira, sendo mais de 50% potencialmente ilegais, por
% ocorrerem dentro em Terra Indígenas e Unidades de Conservação.
% A área de garimpo no bioma cresceu 10x nas últimas três décadas,
% com 301% de expansão em UCs e 495% em TIs.                                    
% -
% A área de garimpos terrestres na bacia do rio Madeira saltaram de
% 3753 ha em 2007 para 9660 ha em 2020, uma expansão de 5907
% hectares (que equivale a mais 8200 campos de futebol).
% -
% A área de garimpos detectados para o ano de 2020 é o recorde
% histórico da série de dados, que conta com 36 anos de imagens de
% satélite.


% O Brasil pertence a um seleto grupo de países capazes de
% desenvolver, operar e utilizar satélites e seus dados. Nessa área o
% país pode se orgulhar da posição que ocupa. Somos um dos países
% que melhor monitora seu território, em diferentes recortes do
% tempo e do espaço, atendendo a diferentes necessidades da
% sociedade civil, da academia ou do mercado financeiro.
% Para tal, fazemos uso de satélites nacionais, como o CBERS-4A e o
% AMAZÔNIA-1 (ambos desenvolvidos pelo INPE), e internacionais,
% sejam públicos (casos do Landsat, da NASA, e do Sentinel, da ESA)
% ou privados (como os nanossatélites da empresa Planet), isso para
% manter curta a lista de exemplos. Somos capazes, portanto, de
% observar e monitorar balsas garimpeiras ao longo de um rio.


% O garimpos ilegal na Amazônia deve ser monitorado, combatido e eliminado.
% -
% O uso de imagens de satélite de alta resolução espacial (como a do CBERS-4A desenvolvido pelo
% INPE) pode ser utilizado para monitorar a dinâmica de balsas garimpeiras na Amazônia e
% auxiliar na fiscalização, combate e controle dos ilícitos ambientais.
% -
% Combinações de múltiplos satélites, de média e alta resolução, do espectro óptico e radar,
% devem ser exploradas para aumentar a frequência de observação e a capacidade de
% monitoramento de balsas garimpeiras, mesmo em condições de intensa nebulosidade.
% -
% Outras inovações tecnológicas, como classificadores de aprendizado profundo (Deep Learning),
% devem ser exploradas para facilitar e automatizar a detecção das balsas garimpeiras e de seus
% impactos associados.




% 3. Método
% Fusão de dados óticos e interpretação visual
% Foi utilizado uma imagem única de 25 de Outubro de 2021, do satélite CBERS-4A, do sensor WPM, com 2 metros
% de resolução após fusão das bandas pan + RGB, para identificar visualmente as balsas garimpeiras no rio Madeira,
% no trecho ao norte de Borba, no Amazonas.
% O CBERS-4A é um satélite nacional, desenvolvido pelo INPE, gratuito e com imagens de todo o território brasileiro
% disponíveis na internet. Trata-se do satélite público de maior resolução espacial do planeta.

% ---



% % % \part{Segunda entrega}
\chapter{Metodologia}
\label{sec:metodologia}
Neste trabalho, a metodologia será seccionada nas seguintes partes:
\begin{itemize}
    \item  Datasets.
    \item  Premissas do problema.
    \item  Proposta de solução.
    \item  Experimento para a solução do problema.

\end{itemize}

Neste capítulo será apresentado em detalhes as premissas do problema de classificação visual bem como a metodologia para sua solução, que será particionada nas conforme proposto a seguir. A seção~\ref{sec:Cap3_Premissas}. apresenta as limitações e condições de contorno do problema. A seção~\ref{sec:Cap3_Dataset} apresenta o conjunto de dados utilizado para o experimento. A seção~\ref{sec:Cap3_Proposta} apresenta a proposta de solução para o problema. A seção~\ref{sec:Cap3_Procedimentos} apresenta os procedimentos do experimento para treino e teste do modelo.

% ----------------------------------------------------------

\section{\textit{Conjunto de dados}}\label{sec:Cap3_Dataset}
Foram utilizados dois conjunto de dados, escolhidos por razões de disponibilidade e representatividade do problema. 

%% Tabelas dos datasets
%- Million A1
%ASGM Ponds Dataset https://zenodo.org/record/6400211
% Planet (URL: https://www.kaggle.com/c/planet-understanding-the-amazon-from-space/data

\subsection{Dataset amazônia do espaço}\label{sec:Cap3_Amazon_dataset}

Este dataset foi publicado em uma competição Kaggle, pela empresa Planet \footnote{https://www.kaggle.com/c/planet-understanding-the-amazon-from-space/data}. Possui resolução de 3m de pixel. Dados foram coletados dos satelites Planet Flock entre 2016 e 2017. Todas imagens são da bacia amazônica. Este dataset concerne ao desmatamento de mangues. Cada amostra é um recorde de 256x256 pixeis RGB, pertencente a 14 classes distintas. Cobrindo condições atmosféricas, coberturas de terreno e fenômenos raros. Foi utilizado em~\cite{9701667} citado no capitulo anterior.

%We utilize a dataset (URL: https://www.kaggle.com/c/planet-understanding-the-amazon-from-space/data) published in a Kaggle competition (by Planet company), containing coarse-resolution imagery data from satellites with varying spatial resolution characteristics, i.e., the imagery has a ground-sample distance (GSD) of 3.7 m and an orthorectified pixel size of 3 m. The data comes from Planet’s Flock tow satellites in both Sun-synchronous and ISS orbits and was collected in the time interval between January 1, 2016, and February 1, 2017. All of the images are derived from the Amazon basin. Mangrove deforestation in the Amazon forest is an intense phenomenon, and a plethora of factors that contribute to deforestation is observed there. Each entry contains imagery data in RGB plus the infrared band in geo-referenced.tiff format. In our experiment, the images are classified into 14 classes and the labels are broken into three groups: atmospheric conditions, common land cover/land use phenomena, and rare land cover/land use phenomena (see Fig. 3). Here, each entry is assigned to one or more classes.


\subsection{Dataset poças de garimpo}\label{sec:Cap2_amazonia_garimpo}

Este dataset foi utilizado em \cite{rs14071746}, mencionado no capitulo anterior. Concerne a tarefa de identificação de mudanças de imagem. Aplicadas a identificação de garimpo artesanal de ouro de pequena escala; Pode ser desafiador de se identificar, dado a variabilidade de condições atmosféricas e baixa resolução. Foram utilizadas imagens de Madre de Dios, região do Peru. Bem como amostras de outros países: Venezuela, Indonesia e Myanmar. (Verificar se esse dataset é utilizável para tarefa de classificação; Resolução parece muito baixa e área muito grande)

% Abstract: Monitoring changes within the land surface and open water bodies is critical for natural resource management, conservation, and environmental policy. While the use of satellite imagery for these purposes is common, fine-scale change detection can be a technical challenge. Difficulties arise from variable atmospheric conditions and the problem of assigning pixels to individual objects. We examined the degree to which two machine learning approaches can better characterize change detection in the context of a current conservation challenge, artisanal small-scale gold mining (ASGM). We obtained Sentinel-2 imagery and consulted with domain experts to construct an open-source labeled land-cover change dataset. The focus of this dataset is the Madre de Dios (MDD) region in Peru, a hotspot of ASGM activity. We also generated datasets of active ASGM areas in other countries (Venezuela, Indonesia, and Myanmar) for out-of-sample testing.


% Dataset Description

%- Understanding the Amazon from Space [20] — multilabel dataset to track the human footprint in the Amazon rainforest; we’ll mostly refer to this as the Amazon dataset.
%WiDS Datathon 2019 [21] — binary dataset for oil palm plantation detection in Borneo; we’ll mostly refer to this as the oil palm dataset.
% Towards Detecting Deforestation [22] — binary dataset for detecting coffee plantations in the Amazon rainforest; we’ll mostly refer to this as the coffee dataset.


% ----------------------------------------------------------

\section{\textit{Premissas}}\label{sec:Cap3_Premissas}

Temos que o problema de identificação em sensoriamento remoto impõe a dificuldade de alta similaridade extra-classes e divergências intra-classes, da qual surge uma dificuldade de generalização e de viés indutivo para identificação de amostras fora das classes treinadas. Temos ainda que o treinamento de tais modelos envolvem um volume massivo de amostras e de custo computacional. 


% ----------------------------------------------------------

\section{\textit{Proposta de solução}}\label{sec:Cap3_Proposta}

Como solução para o problema, foi proposta a utilização de um modelo ViT pré-treinado com um dataset expressivo e realizar o fine-tune para o conjunto de dados de interesse. Dessa forma, obtendo um modelo com boa capacidade de generalização e com menor custo computacional para ser treinado.

Para solucionar o problema de generalização e dataset limitado da região de interesse, propomos a utilização de um modelo pré-treinado explicado em~\ref{sec:Cap2_transfer} um dataset extensivo de imagens aéreas e de satélites, de forma a aproveitar seu extrator de características, como é realizado em~\cite{wang2022empirical}. Assim o modelo será re-treinado (\textit{fine tunning}) para a região de interesse mantendo os pesos das camadas de encoder otimizar apenas a camada MLP fortemente conectadas, assim como a camadas de saída softmax. 

Dessa forma, a arquitetura proposta é o ViT apresentado em~\cite{wang2022empirical}
composta por camadas hierarquicas de encoders transformers  que funciona como extrator de características. As camadas seguintes são camadas totalmente conectadas seguidas por uma camada \textit{softmax} que realiza a classificação.
% ----------------------------------------------------------

\section{\textit{Ambiente e ferramentas}}\label{sec:Cap3_Ferramentas}


O Ambiente dos experimentos será em cadernos \textit{Jupyters}, para ser facilmente replicável e ser executável em nuvem, com a possibilidade de alugar recursos computacionais no Google Collab. Também será utilizado o \textit{framework PyTorch}, por razões de disponibilidade de métodos e conhecimentos do autor.

\subsection{\textit{PyTorch}}\label{sec:Cap2_PyTorch}
Pytorch é um framework código aberto de Python para aprendizado de máquinas. É implementado em C++ e CUDA para otimizações de computações numéricas e matriciais, que são extensivas para esta aplicação.
Foi desenvolvido pelo Facebook e é atualmente amplamente adotado pela comunidade de pesquisa e mercado de aprendizado de máquina. Possui muitas implementações das ferramentas mais utilizadas, especificamente aplicadas a \textit{Deep Learning} e visão computacional. Tem ampla adoção devido à intenção de ser um framework de fácil uso e alto nível, com muitas abstrações e técnicas já implementadas.


% ----------------------------------------------------------


% TODO: citar o paper de aplicação direta
\section{\textit{Experimentos}}\label{sec:Cap3_Experimentos}

Para a construção da nossa solução desejada, experimentaremos combinar um modelo pré-treinado com camadas internas e de saída treinadas para a região de interesse.
Consiste em fazer experimentos em uma complexidade crescente, e replicando resultados para garantir corretude. Será simplificado a reprodução para apenas um dataset.


Para obter o modelo proposto e de melhor desempenho, foi proposto a seguinte sequência de experimentos para construir o modelo final:

\begin{enumerate}
\item  Replicar experimentos dos trabalhos \ref{sec:Cap2_million} com checkpoints fornecidos. 
\item  Replicar experimentos de \ref{sec:Cap2_ForestViT} em um modelo base de comparação baseado em ResNet-50
\item  Replicar o experimento do \ref{sec:Cap2_ForestViT} utilizando um modelo ViT pré treinado para sensoriamento remoto
\item Avaliar desempenho, comparar com os experimentos e trabalhos bases
\end{enumerate}


    
% ----------------------------------------------------------

\section{\textit{Procedimentos}}\label{sec:Cap3_Procedimentos}

Os procedimentos do experimento principal consiste principalmente nas etapas de 
pré-processamento, treino e validação. Explicadas a diante.

\subsection{\textit{Pré-processamento}}\label{sec:Cap3_PreProcess}
O pré-processamento consistem em preparar as amostras do dataset de interesse para treino e validação. Para o dataset floresta amazônica temos 40000 amostras de treino e 4000 de teste. Cada recorte de resolução 256x256 px é aplicado um \textit{downsampling} para a resolução de entrada do modelo pré-treinado, de 224 × 224. 

\subsection{\textit{Treino}}\label{sec:Cap3_Treino}
Para a etapa de treino, utilizamos a arquitetura proposta em~\ref{sec:Cap3_Proposta}, que consiste de um modelo pré-treinado, e retreinando suas ultimas camadas de classificação por camadas fortemente conectadas seguidas de camadas de softmax, como ilustra a imagem a seguir:


TODO Imagem da arquitetura proposta.



\subsection{\textit{Validação}}\label{sec:Cap3_Validacao}

A etapa de validação será realizada a medida do desempenho do modelo no conjunto de testes para classificação.




% Chip (Image) Data Format

% The chips for this competition were derived from Planet's full-frame analytic scene products using our 4-band satellites in sun-synchronous orbit (SSO) and International Space Station (ISS) orbit. The set of chips for this competition use the GeoTiff format and each contain four bands of data: red, green, blue, and near infrared. The specific spectral response of the satellites can be found in the Planet documentation. Each of these channels is in 16-bit digital number format, and meets the specification of the Planet four band analytic ortho scene product.

% For purposes of the competition we have stripped out all of the geotiff information regarding the chip footprint and ground control points (GCPs). The imagery has a ground-sample distance (GSD) of 3.7m and an orthorectified pixel size of 3m. The data comes from Planet's Flock 2 satellites in both sun-synchronous and ISS orbits and was collected between January 1, 2016 and February 1, 2017. All of the scenes come from the Amazon basin which includes Brazil, Peru, Uruguay, Colombia, Venezuela, Guyana, Bolivia, and Ecuador (see map below).

% We have also included a set of JPG chips for reference and practice. These chips were processed using the Planet visual product processor and then saved as jpg chips. These chips are provided as a reference to the scene content, but we expect that the additional information in the tif chips will be more fruitful for the competition.

% https://commons.wikimedia.org/w/index.php?curid=4745680 Above: A map of the Amazon basin.
% Labeling Process and Data Quality

% enter image description here

% To assemble this data set we set out with an initial specification of the phenomena we wished to find and include in the final data set. From that initial specification we created a "wish list" of scenes where we included a ballpark number of scenes required to get a sufficient number of chips to demonstrate the phenomena. This initial set of scenes was painstakingly collected by our Berlin team using Planet Explorer. All told this initial set of scenes numbered approximately 1600 and covered a land area of thirty million hectares.

% This initial set of scenes was then processed using a custom product processor to create the jpg and 4-band tif chips. Any chip that did not have a full and complete four band product was omitted. This initial set of over 150,000 chips was then divided into two sets, a "hard" and an "easy" set. The easy set contained scenes that the Berlin team identified as having easier-to-identify labels like primary rainforest, agriculture, habitation, roads, water, and cloud conditions. The harder set of data was derived from scenes where the Berlin team had selected for shifting cultivation, slash and burn agriculture, blow down, mining, and other phenomenon.

% The chips were labeled using the Crowd Flower platform and a mixture of crowd-sourced labor and our Berlin and San Francisco teams. While the utmost care was taken to get a large and well-labeled dataset, we are aware that not all of the labels in our dataset are correct. Governments around the world retain a large number of highly trained analysts to review images and even they can't always agree on what is present in a given satellite image.

% Moreover, the commonly prescribed approach for labeling data in the GIS community is to use actual ground truth data to label scenes, which is both costly and time consuming. With this in mind we do believe our data has a reasonably high signal to noise ratio and is sufficient for training. Given the ease and expediency of crowd labeling, we believe that a large, relatively inexpensive and rapidly labeled dataset is better than a small, more definitive but less diverse dataset. We are interested to see how competitors handle any inaccuracies.
% Class Labels

% The class labels for this task were chosen in collaboration with Planet's Impact team and represent a reasonable subset of phenomena of interest in the Amazon basin. The labels can broadly be broken into three groups: atmospheric conditions, common land cover/land use phenomena, and rare land cover/land use phenomena. Each chip will have one and potentially more than one atmospheric label and zero or more common and rare labels. Chips that are labeled as cloudy should have no other labels, but there may be labeling errors. Sample chips with labels

% Above: Sample chips and their labels.

% As discussed in the data collection portion of this document, the chip labels are inherently noisy due to the labeling process and ambiguity of features, and scenes may either omit class labels or have incorrect class labels. Part of the challenge of this competition is to figure out how to work with noisy data.
% Cloud Cover Labels

% Clouds are a major challenge for passive satellite imaging, and daily cloud cover and rain showers in the Amazon basin can significantly complicate monitoring in the area. For this reason we have chosen to include a cloud cover label for each chip. These labels closely mirror what one would see in a local weather forecast: clear, partly cloudy, cloudy, and haze. For our purposes haze is defined as any chip where atmospheric clouds are visible but they are not so opaque as to obscure the ground. Clear scenes show no evidence of clouds, and partly cloudy scenes can show opaque cloud cover over any portion of the image. Cloudy images have 90% of the chip obscured by opaque cloud cover.
% Examples of Cloudy Scenes

% Cloudy Scene enter image description here
% Examples of Partly Cloudy Scenes

% Partly Cloudy Scene Partly Cloudy Scene
% Examples of Hazy Scenes

% Partly Cloudy Scene Partly Cloudy Scene
% More Common Labels

% The common labels in this data set are rainforest, agriculture, rivers, towns/cities, and roads. Examples of each class are given below.
% Primary Rain Forest

% The overwhelming majority of the data set is labeled as "primary", which is shorthand for primary rainforest, or what is known colloquially as virgin forest. Generally speaking, the "primary" label was used for any area that exhibited dense tree cover.This Mongobay article gives a concise description of the difference between primary and secondary rainforest, but distinguishing between the two is difficult solely using satellite imagery. This is particularly true in older "secondary" forests. Primary Rainforest

% Above: Approximately 25,000 acres of untouched primary rainforest.
% Water (Rivers & Lakes)

% Rivers, reservoirs, and oxbow lakes are important features of the Amazon basin, and we used the water tag as a catch-all term for these features. Rivers in the Amazon basin often change course and serve as highways deep into the forest. The changing course of these rivers creates new habitat but can also strand endangered Amazon River Dolphins. River

% Above: A larger and slower river with significant sand bars. The brown color comes from significant silt deposits. River

% Above: A small tributary joins a larger river system. The deep brown color of the river is noticeable near the bright sand bars.
% Habitation

% The habitation class label was used for chips that appeared to contain human homes or buildings. This includes anything from dense urban centers to rural villages along the banks of rivers. Small, single-dwelling habitations are often difficult to spot but usually appear as clumps of a few pixels that are bright white. Habitation

% Above: A larger city in the Amazon basin. Habitation

% Above: A large city.
% Agriculture

% Commercial agriculture, while an important industry, is also a major driver of deforestation in the Amazon. For the purposes of this dataset, agriculture is considered to be any land cleared of trees that is being used for agriculture or range land.

% More reading on agriculture in the Amazon:

%     Sugarcane in Bolivia
%     Papaya cultivation destroying Peruvian Rainforest
%     Harvests in Rio Grande do Sul

% Agriculture 1

% Above: An agricultural area that showing the end state of "fishbone" deforestation. Agriculture 2

% Above: A newer agricultural area showing "fishbone" deforestation.
% Road

% Roads are important for transportation in the Amazon but they also serve as drivers of deforestation. In particular, "fishbone" deforestation often follows new road construction, while smaller logging roads drive selective logging operations. For our data, all types of roads are labeled with a single "road" label. Some rivers look very similar to smaller logging roads, and consequently there may be some noise in this label. Analysis of the image using the near infrared band may prove useful in disambiguating the two classes.

% More information: - Roads in the Amazon - NASA article on Fishbone Deforestation

% Road 1

% Above: classic "Fishbone" deforestation following a road. Road 2

% Above: roads snake out of a small town in the Amazon.
% Cultivation

% Shifting cultivation is a subset of agriculture that is very easy to see from space, and occurs in rural areas where individuals and families maintain farm plots for subsistence. This article by MongaBay by MongaBay gives a detailed overview of the practice. This type of agriculture is often found near smaller villages along major rivers, and at the outskirts of agricultural areas. It typically relies on non-mechanized labor, and covers relatively small areas. cultivation

% Above: A zoomed-in area showing cultivation (right side of river) cultivation

% Above: A zoomed-in area showing cultivation and some selective logging. Dark areas indicate recent slash/burn activity
% Bare Ground

% Bare ground is a catch-all term used for naturally occuring tree free areas that aren't the result of human activity. Some of these areas occur naturally in the Amazon, while others may be the result from the source scenes containing small regions of biome much similar to the pantanal or cerrado. bare ground

% Above: a naturally occuring bare area in the cerrado. bare ground

% Above: a naturally occuring bare area in the cerrado.
% Less Common Labels
% Slash and Burn

% Slash-and-burn agriculture can be considered to be a subset of the shifting cultivation label and is used for areas that demonstrate recent burn events. This is to say that the shifting cultivation patches appear to have dark brown or black areas consistent with recent burning.This NASA Earth Observatory article gives a good primer on the practice as does this wikipedia article. Above: ground view of slash and burn agriculture. By Alzenir Ferreira de Souza slash burn

% Above: A zoomed-in view of an area with shifting cultivation with evidence of a recent fire. slash burn

% Above: A zoomed-in view of an area with shifting cultivation and evidence of a recent fire.
% Selective Logging

% The selective logging label is used to cover the practice of selectively removing high value tree species from the rainforest (such as teak and mahogany). From space this appears as winding dirt roads adjacent to bare brown patches in otherwise primary rain forest. This Mongabay Article covers the details of this process. Global Forest Watch is another great resource for learning about deforestation and logging. logging

% Above: The brown lines on the right of this scene are a logging road. Note the small brown dots in the area around the road. logging

% Above: A zoomed image of logging roads and selective logging. logging

% Above: A zoomed image of logging roads and selective logging.
% Blooming

% Blooming is a natural phenomenon found in the Amazon where particular species of flowering trees bloom, fruit, and flower at the same time to maximize the chances of cross pollination. These trees are quite large and these events can be seen from space. Planet recently captured a similar event in Panama. bloom

% Above: a zoomed and contrast enhanced of a bloom event in the Amazon basin. The red arrows point to a few specific trees. The canopies of these trees can be over 30m across (~100ft).
% Conventional Mining

% There are a number of large conventional mines in the Amazon basin and the number is steadily growning. This label is used to classify large-scale legal mining operations.

% mine

% Above: A conventional mine in the Amazon.
% "Artisinal" Mining

% Artisinal mining is a catch-all term for small scale mining operations. Throughout the Amazon, especially at the foothills of the Andes, gold deposits lace the deep, clay soils. Artisanal miners, sometimes working illegally in land designated for conservation, slash through the forest and excavate deep pits near rivers. They pump a mud-water slurry into the river banks, blasting them away so that they can be processed further with mercury - which is used to separate out the gold. The denuded moonscape left behind takes centuries to recover.

%     Illegal and artisanal mines in Peru
%     Images of artisanal mining in Peru
%     MAAP Amazon Report #36
%     MAAP Amazon Report #49
%     Global Forest Watch article on Mining artisanal mine

% Above: A zoomed image of an artisanal mine in Peru. artisanal mine

% Above: A zoomed image of an artisanal mine in Peru.
% Blow Down

% Blow down, also called windthrow, is a naturally occurring phenomenon in the Amazon. Briefly, blow down events occur during microbursts where cold dry air from the Andes settles on top of warm moist air in the rainforest. The colder air punches a hole in the moist warm layer, and sinks down with incredible force and high speed (in excess of 100MPH). These high winds topple the larger rainforest trees, and the resulting open areas are visible from space. The open areas do not stay visible for along as plants in the understory rush in to take advantage of the sunlight.

%     MAAP #55: Blow Down Report in Peru Detailed
%     MAAP #54: Blow Down Report in Peru
%     National Geographic Article on Blow Down
%     Nature article on the size and frequency of blow down events. blow down

% Above: A recent blow down event in the Amazon circled in red. Note the light green of the forest understory and the pattern of tree loss.





\label{sec:Resultados}
\chapter{Resultados}
Tentativas de reproduzir os todos os experimentos dos trabalhos mencionados não foram bem sucedidos, por falta de reproducibilidade, contudo deram uma direção de como seria a metodologia de realização e validação dos experimentos.

Tentativa inicial:

\begin{enumerate}
    \item  Replicar experimentos dos trabalhos~\ref{sec:Cap2_million} com \textit{checkpoints}\footnote{Captura dos pesos de uma rede a partir de certo ponto do processo de treino} disponibilizados. 
    \item  Replicar experimentos de \ref{sec:Cap2_ForestViT} em um modelo base de comparação baseado em em uma CNN ResNet-50
    \item  Replicar o experimento do~\ref{sec:Cap2_ForestViT} utilizando um modelo ViT pré treinado para sensoriamento remoto
    \item Avaliar desempenho, comparar com os experimentos e trabalhos bases \ref{sec:Cap2_revisao_literatura}
    \end{enumerate}

O segundo item foi possível de ser realizado. Realizando \textit{fine-tune} do modelo o modelo Resnet-16 inicialmente. Foram utilizados os pesos dessa mesma rede treinada para o dataset IMAGENET-1k, realizando a troca das camadas de saída, originalmente para 1000 classes. Foram Removidas e adicionados uma camada completamente conectada de entrada igual ao numero de neurônios da penultima camada. Após a ultima camada foi adicionado uma camada de sigmoid, que realizará a conversão de valores lineares para probabilidade de cada classe.

O modelo obteve um desempenho inicialmente satisfatório de score f2 de 0.88.

A partir deste modelo inicial, foram feitos vários ajustes de hiperparâmetros e de componentes a fim de aperfeiçoar o desempenho da rede. Dentre as melhorias, constam:

\begin{enumerate}
    \item  Aumentar capacidade da rede, Adicionando mais camadas de saída
    \item  Aumentar capacidade da rede, passando para o modelo Resnet-50
    \item Adicionar regularização de decaimento de pesos
    \item Função de perda entropia cruzada binária
    \item Função de perda entropia cruzada binária com pesos 
    \item Função de perda entropia cruzada binária Focal
    \item Otimizador gradiente descendente estocástico
    \item Otimizador adaptativo adam 
    \item Otimizador adam com decaimento de pesos
    \item Varredura de diferentes taxas de aprendizado
    \item Agendamento da taxa de aprendizado
    \item Transferência de aprendizado vs Fine Tune
    \item Data augmentation aleatória
    \item Amostrador aleatório com probabilidades 
    \item  Replicar experimentos de \ref{sec:Cap2_ForestViT} em um modelo base de comparação baseado em em uma CNN ResNet-50
    \item  Replicar o experimento do~\ref{sec:Cap2_ForestViT} utilizando um modelo ViT pré treinado para sensoriamento remoto
    \item Avaliar desempenho, comparar com os experimentos e trabalhos bases \ref{sec:Cap2_revisao_literatura}
    \end{enumerate}





% ----------------------------------------------------------
% Finaliza a parte no bookmark do PDF
% para que se inicie o bookmark na raiz
% e adiciona espaço de parte no Sumário
% ----------------------------------------------------------
\phantompart

% ---
% Conclusão
% ---
\chapter{Conclusão}
Neste trabalho foi possível aplicar métodos modernos de visão computacional e aprendizado profundo em uma prova de conceito de uma aplicação atual de sensoriamento remoto e de defesa do meio ambiente.  Teve como ponto de partida um dataset já estudado e uma busca de metodologia para problemas análogos. Obteve-se um modelo baseado em transformers visuais superior aos já estabelecidos redes convolucionais resuduais.

\section{Trabalhos futuros}

Para trabalhos futuros, é possível aplicar a mesma metodologia para avaliar a comparação em conjunto de dados diferentes. Algumas opções de experimentos não foram exploradas, como:
\begin{itemize}
    \item Embutir esta prova de conceito em uma aplicação real para monitoramento remoto.
    \item Utilizar função de perda Hamming, adequada para classificações multi-rótulos.
    \item Utilizar Aumento de dado, por meio de geração de dados sintéticos para classes raras utilizando ruido gaussiano.
    \item Utilizar o canal de infravermelho próximo, já que vários satélites provém esse espectro, e várias técnicas de sensoriamento remoto utilizam desse espectro e de ondas mais longas.
    \item Normalizar cadas amostra com a média e desvio do próprio dataset, em vez de utilizar os do conjunto de dados ImageNet1k.
\end{itemize}

% ---

% ----------------------------------------------------------
% ELEMENTOS PÓS-TEXTUAIS
% ----------------------------------------------------------
\postextual
% ----------------------------------------------------------

% ----------------------------------------------------------
% Referências bibliográficas
% ----------------------------------------------------------
\bibliography{Referencias}


% % ---
% % Inicia os anexos
% % ---
\begin{anexosenv}

% % Imprime uma página indicando o início dos anexos
\partanexos
\chapter{Blocos Adicionais}
\label{blocosadicionais}

Alguns circuitos adicionais não foram diretamente especificados ao longo deste trabalho, por serem considerados pelo autor circuitos mais simples e com ampla flexibilidade de escolha de seus par\^ametros. Os projetos se deram de acordo com aproveitamento de circuitos j\'a existentes advindos de outros trabalhos disponibilizados pelo orientador ou então previamente desenvolvidos pelo autor.

\renewcommand{\NomeBloco}{\emph{Inversor}}
\renewcommand{\NomeBlocoNoIt}{Inversor}
\renewcommand{\NomePTab}{tab_\NomeBlocoNoIt}
\renewcommand{\NomeSTab}{tab_\NomeBlocoNoIt2}
\renewcommand{\NomePFig}{fig_\NomeBlocoNoIt}
\renewcommand{\NomeSFig}{fig_\NomeBlocoNoIt2}
\renewcommand{\NomeTTab}{tab_\NomeBlocoNoIt3}

\section{Inversor}
\label{inversor1}

O bloco \NomeBloco{} tem a finalidade de receber uma entrada digital, e colocar o n\'ivel l\'ogico invertido na sa\'ida. A \autoref{\NomePTab} indica a Tabela Verdade do bloco.

\begin{table}[htbp]

\caption{Tabela Verdade do bloco \NomeBloco}%
\label{\NomePTab}
\centering
\begin{tabular}{cc}
    \toprule
    Entrada & Saída \\
    \midrule \midrule
    0 & 1 \\
    \midrule
    1 & 0 \\
\bottomrule

\end{tabular}
\fonte{Produzido pelo autor.}
\end{table}

O bloco apresenta as definições de sinais de entrada e sa\'ida referidos na \autoref{\NomeSTab}.

\begin{table}[htbp]
\caption{Sinais do bloco \NomeBloco}
\label{\NomeSTab}
\centering
\begin{tabular}{ccl}

    \toprule
    Sinal & Tipo    & Descrição        \\
    \midrule \midrule
    In    & Entrada & Sinal de Entrada \\
    \midrule
    Out   & Saída   & Sinal de Sa\'ida   \\
    \bottomrule
\end{tabular}
\legend{Fonte: Produzido pelo autor}
\end{table}

O circuito projetado para o bloco \'e demonstrado na \autoref{\NomePFig}.

\begin{figure}[htb]
  \begin{minipage}{0.4\textwidth}
    \centering
    \caption{\label{\NomePFig}Circuito CMOS projetado para o bloco \NomeBloco}
    \includegraphics[scale=0.3]{Circuitos/NOT.png}
    \legend{Fonte: Produzido pelo autor}
  \end{minipage}
  \hfill
  \begin{minipage}{0.4\textwidth}
    \centering
    \caption{\label{\NomeSFig}Representação em bloco do \NomeBloco}
    \includegraphics[scale=0.3]{Circuitos/NOT_block.png}
    \legend{Fonte: Produzido pelo autor}
  \end{minipage}
\end{figure}

Os transistores utilizados no bloco \NomeBloco{} apresentam os par\^ametros mostrados na \autoref{\NomeTTab}.

\begin{table}[htbp]
\caption{Transistores do Bloco \NomeBloco}
\label{\NomeTTab}
\centering
\begin{tabular}{ccccc}
\toprule
Transistor & W ($\mu$m)  & L ($\mu$m)           & M (n° dispositivos) & S (n° dispositivos)\\
\midrule \midrule
Q1 & 1,2 & 0,18 & 1 & 1\\
\midrule
Q2 & 0,6 & 0,18 & 1 & 1\\
\bottomrule
\end{tabular}
\legend{Fonte: Produzido pelo autor}
\end{table}
\renewcommand{\NomeBloco}{\emph{NAND}}
\renewcommand{\NomeBlocoNoIt}{NAND}
\renewcommand{\NomePTab}{tab_\NomeBlocoNoIt}
\renewcommand{\NomeSTab}{tab_\NomeBlocoNoIt2}
\renewcommand{\NomePFig}{fig_\NomeBlocoNoIt}
\renewcommand{\NomeSFig}{fig_\NomeBlocoNoIt2}
\renewcommand{\NomeTTab}{tab_\NomeBlocoNoIt3}

\section{NAND}

O bloco \NomeBloco{} tem a finalidade de receber duas entradas digitais, e rcolocar o resultado da operação NAND em sua sa\'ida. A \autoref{\NomePTab} indica a Tabela Verdade do bloco.

\begin{table}[htbp]

\caption{Tabela Verdade do bloco \NomeBloco}%
\label{\NomePTab}
\centering
\begin{tabular}{ccc}
\toprule
    A & B & Out \\
    \midrule \midrule
    0 & 0 & 1 \\
    \midrule
    0 & 1 & 1\\
    \midrule
    1 & 0 & 1\\
    \midrule
    1 & 1 & 0\\
\bottomrule

\end{tabular}
\fonte{Produzido pelo autor.}
\end{table}

O bloco apresenta as definições de sinais de entrada e sa\'ida referidos na \autoref{\NomeSTab}.

\begin{table}[htbp]
\caption{Sinais do bloco \NomeBloco}
\label{\NomeSTab}
\centering
\begin{tabular}{ccl}

    \toprule
    Sinal & Tipo    & Descrição        \\
    \midrule \midrule
    A    & Entrada & Sinal de Entrada A \\
    \midrule
    B    & Entrada & Sinal de Entrada B \\
    \midrule
    Out    & Sa\'ida & Sinal de sa\'ida \\
    \bottomrule
\end{tabular}
\legend{Fonte: Produzido pelo autor}
\end{table}

O circuito projetado para o bloco \'e demonstrado na \autoref{\NomePFig}.

\begin{figure}[htbp]
 \centering
  \begin{minipage}{0.4\textwidth}
    \centering
    \caption{\label{\NomePFig}Circuito CMOS projetado para o bloco \NomeBloco}
    \includegraphics[scale=0.3]{Circuitos/NAND.png}
    \legend{Fonte: Produzido pelo autor}
  \end{minipage}
  \hfill
  \begin{minipage}{0.4\textwidth}
    \centering
    \caption{\label{\NomeSFig}Representação em bloco do \NomeBlocoNoIt} 
    \includegraphics[scale=0.3]{Circuitos/NAND_block.png}
    \legend{Fonte: Produzido pelo autor}
  \end{minipage}
\end{figure}
\renewcommand{\NomeBloco}{\emph{Decoder 2x4}}
\renewcommand{\NomeBlocoNoIt}{Decoder 2x4}
\renewcommand{\NomePTab}{tab_\NomeBlocoNoIt}
\renewcommand{\NomeSTab}{tab_\NomeBlocoNoIt2}
\renewcommand{\NomePFig}{fig_\NomeBlocoNoIt}
\renewcommand{\NomeSFig}{fig_\NomeBlocoNoIt2}
\renewcommand{\NomeTTab}{tab_\NomeBlocoNoIt3}

\section{Decoder 2x4}

O bloco \NomeBloco{}\footnote{Circuito disponibilizado por Dalton Martini Colombo, orientador do trabalho aqui apresentado} tem a função de colocar um bit '1' em uma das sa\'idas, enquanto todas outras são iguais a '0'. A \autoref{\NomePTab} indica a Tabela Verdade do bloco.

\begin{table}[htbp]

\caption{Tabela Verdade do bloco \NomeBloco}%
\label{\NomePTab}
\centering
\begin{tabular}{cccccc}
    \toprule
    In0 & In1 & Out0 & Out1 & Out2 & Out3 \\
    \midrule \midrule
    0 & 0 & 1 & 0 & 0 & 0 \\
    \midrule
    0 & 1 & 0 & 1 & 0 & 0 \\
    \midrule
    1 & 0 & 0 & 0 & 1 & 0 \\
    \midrule
    1 & 1 & 0 & 0 & 0 & 1 \\
\bottomrule

\end{tabular}
\fonte{Produzido pelo autor.}
\end{table}

O bloco apresenta as definições de sinais de entrada e sa\'ida referidos na \autoref{\NomeSTab}.

\begin{table}[htbp]
\caption{Sinais do bloco \NomeBloco}
\label{\NomeSTab}
\centering
\begin{tabular}{ccl}

    \toprule
    Sinal & Tipo    & Descrição        \\
    \midrule \midrule
    In1    & Entrada & Primeira entrada de seleção de sa\'ida \\
    \midrule
    In2    & Entrada & Segunda entrada de seleção de sa\'ida \\
    \midrule
    Out0 & Sa\'ida & Sa\'ida 0\\
    \midrule
    Out1 & Sa\'ida & Sa\'ida 1\\
    \midrule
    Out2 & Sa\'ida & Sa\'ida 2\\
    \midrule
    Out3 & Sa\'ida & Sa\'ida 3\\
    \bottomrule
\end{tabular}
\legend{Fonte: Produzido pelo autor}
\end{table}

O circuito projetado para o bloco \'e demonstrado na \autoref{\NomePFig}.

\begin{figure}[htbp]
 \centering
    \centering
    \caption{\label{\NomePFig}Circuito CMOS projetado para o bloco \NomeBloco}
    \includegraphics[scale=0.3]{Circuitos/decoder.png}
    \legend{Fonte: Produzido pelo autor}
\end{figure}

\begin{figure}[htbp]
 \centering
    \centering
    \caption{\label{\NomeSFig}Representação em bloco do \NomeBloco} 
    \includegraphics[scale=0.3]{Circuitos/decoder_block.png}
    \legend{Fonte: Produzido pelo autor}
\end{figure}
\renewcommand{\NomeBloco}{\emph{Seletor 4x1}}
\renewcommand{\NomeBlocoNoIt}{Seletor 4x1}
\renewcommand{\NomePTab}{tab_\NomeBlocoNoIt}
\renewcommand{\NomeSTab}{tab_\NomeBlocoNoIt2}
\renewcommand{\NomePFig}{fig_\NomeBlocoNoIt}
\renewcommand{\NomeSFig}{fig_\NomeBlocoNoIt2}
\renewcommand{\NomeTTab}{tab_\NomeBlocoNoIt3}

\section{Seletor 4x1}

O bloco \NomeBloco{}\footnote{Circuito disponibilizado por Dalton Martini Colombo, orientador do trabalho aqui apresentado} tem a finalidade de selecionar uma entrada, e a colocar na sa\'ida. A entrada \'e selecionada atrav\'es de 4 entradas de seleção, em que apenas uma deve estar em n\'ivel l\'ogico um por vez. A \autoref{\NomePTab} indica a Tabela Verdade do bloco. Embora tenha uma l\'ogica digital, o circuito permite entradas e sa\'idas anal\'ogicas.

\begin{table}[htbp]

\caption{Tabela Verdade do bloco \NomeBloco}%
\label{\NomePTab}
\centering
\begin{tabular}{ccccc}
    \toprule
    Sel1 & Sel2 & Sel3 & Sel4 & Out \\
    \midrule \midrule
    1 & 0 & 0 & 0 & In1 \\
    \midrule
    0 & 1 & 0 & 0 & In2 \\
    \midrule
    0 & 0 & 1 & 0 & In3 \\
    \midrule
    0 & 0 & 0 & 1 & In4 \\
    \midrule
    0 & 0 & 0 & 0 & Z \\
    \midrule
    X & X & X & X & X \\
\bottomrule

\end{tabular}
\fonte{Produzido pelo autor.}
\end{table}

O bloco apresenta as definições de sinais de entrada e sa\'ida referidos na \autoref{\NomeSTab}.

\begin{table}[htbp]
\caption{Sinais do bloco \NomeBloco}
\label{\NomeSTab}
\centering
\begin{tabular}{ccl}

    \toprule
    Sinal & Tipo    & Descrição        \\
    \midrule \midrule
    In1    & Entrada & Primeira entrada \\
    \midrule
    Sel1    & Entrada & Seleção de entrada In1 \\
    \midrule
    In2    & Entrada & Segunda entrada \\
    \midrule
    Sel2    & Entrada & Seleção de entrada In2 \\
    \midrule
    In3    & Entrada & Terceira entrada \\
    \midrule
    Sel3    & Entrada & Seleção de entrada In3 \\
    \midrule
    In4    & Entrada & Quarta entrada \\
    \midrule
    Sel4    & Entrada & Seleção de entrada In4 \\
    \midrule
    Out   & Saída   & Sinal de Sa\'ida selecionado   \\
    \bottomrule
\end{tabular}
\legend{Fonte: Produzido pelo autor}
\end{table}

O circuito projetado para o bloco \'e demonstrado na \autoref{\NomePFig}.

\begin{figure}[htb]
 \centering
    \centering
    \caption{Circuito CMOS projetado para o bloco \NomeBloco} \label{\NomePFig}
    \includegraphics[scale=0.3]{Circuitos/sel4x1.png}
    \legend{Fonte: Produzido pelo autor}
\end{figure}

\begin{figure}[htb]
    \centering
    \caption{\label{\NomeSFig}Representação em bloco do \NomeBloco}
    \includegraphics[scale=0.3]{Circuitos/sel4x1_block.png}
    \legend{Fonte: Produzido pelo autor}
\end{figure}

Os transistores utilizados no bloco \NomeBloco{} apresentam os par\^ametros mostrados na \autoref{\NomeTTab}.

\begin{table}[htbp]
\caption{Transistores do Bloco \NomeBloco}
\label{\NomeTTab}
\centering
\begin{tabular}{ccccc}
\toprule
Transistor & W ($\mu$m)  & L ($\mu$m)           & M (n° dispositivos) & S (n° dispositivos)\\
\midrule \midrule
Q1 & 1,2 & 0,18 & 1 & 1\\
\midrule
Q2 & 0,6 & 0,18 & 1 & 1\\
\bottomrule
\end{tabular}
\legend{Fonte: Produzido pelo autor}
\end{table}
\renewcommand{\NomeBloco}{\emph{Multiplexador 7x1}}
\renewcommand{\NomeBlocoNoIt}{Multiplexador 7x1}
\renewcommand{\NomePTab}{tab_\NomeBlocoNoIt}
\renewcommand{\NomeSTab}{tab_\NomeBlocoNoIt2}
\renewcommand{\NomePFig}{fig_\NomeBlocoNoIt}
\renewcommand{\NomeSFig}{fig_\NomeBlocoNoIt2}
\renewcommand{\NomeTTab}{tab_\NomeBlocoNoIt3}

\section{Multiplexador 7x1}

O bloco \NomeBloco{}\footnote{Circuito disponibilizado por Dalton Martini Colombo, orientador do trabalho aqui apresentado} tem a função de colocar na sa\'ida o valor correspondente \'a entrada selecionada. Nesse bloco, 7 entradas distintas podem ser selecionadas. A \autoref{\NomePTab} indica a Tabela Verdade do bloco. Embora tenha uma l\'ogica digital, o circuito permite entradas e sa\'idas anal\'ogicas.

\begin{table}[htbp]

\caption{Tabela Verdade do bloco \NomeBloco}%
\label{\NomePTab}
\centering
\begin{tabular}{ccccc}
    \toprule
    D0 & D1 & D2 & D3 & Out \\
    \midrule \midrule
    0 & 0 & 0 & 0 & In0 \\
    \midrule
    0 & 1 & 0 & 0 & In1 \\
    \midrule
    1 & 0 & 0 & 0 & In2 \\
    \midrule
    1 & 1 & 0 & 0 & In3 \\
    \midrule
    X & X & 0 & 1 & In4 \\
    \midrule
    X & X & 1 & 0 & In5 \\
    \midrule
    X & X & 1 & 1 & In6 \\
\bottomrule

\end{tabular}
\fonte{Produzido pelo autor.}
\end{table}

O bloco apresenta as definições de sinais de entrada e sa\'ida referidos na \autoref{\NomeSTab}.

\begin{table}[htbp]
\caption{Sinais do bloco \NomeBloco}
\label{\NomeSTab}
\centering
\begin{tabular}{ccl}

    \toprule
    Sinal & Tipo    & Descrição        \\
    \midrule \midrule
    D0    & Entrada & Primeira entrada de seleção de sa\'ida \\
    \midrule
    D1    & Entrada & Segunda entrada de seleção de sa\'ida \\
    \midrule
    D2    & Entrada & Segunda entrada de seleção de sa\'ida \\
    \midrule
    D3    & Entrada & Segunda entrada de seleção de sa\'ida \\
    \midrule
    Vout & Sa\'ida & Sa\'ida do Multiplexador\\
    \midrule
    \bottomrule
\end{tabular}
\legend{Fonte: Produzido pelo autor}
\end{table}

O circuito projetado para o bloco \'e demonstrado na \autoref{\NomePFig}.

\begin{figure}[htbp]
 \centering
    \centering
    \caption{\label{\NomePFig}Circuito CMOS projetado para o bloco \NomeBloco}
    \includegraphics[scale=0.3]{Circuitos/mux7x1.png}
    \legend{Fonte: Produzido pelo autor}
\end{figure}

\begin{figure}[htbp]
 \centering
    \centering
    \caption{\label{\NomeSFig}Representação em bloco do \NomeBloco} 
    \includegraphics[scale=0.3]{Circuitos/mux7x1_block.png}
    \legend{Fonte: Produzido pelo autor}
\end{figure}
\renewcommand{\NomeBloco}{\emph{Porta de Transmissão}}
\renewcommand{\NomeBlocoNoIt}{Porta de Transmissão}
\renewcommand{\NomePTab}{tab_\NomeBlocoNoIt}
\renewcommand{\NomeSTab}{tab_\NomeBlocoNoIt2}
\renewcommand{\NomePFig}{fig_\NomeBlocoNoIt}
\renewcommand{\NomeSFig}{fig_\NomeBlocoNoIt2}
\renewcommand{\NomeTTab}{tab_\NomeBlocoNoIt3}

\section{Porta de Transmissão}
\label{portatg}

O bloco \emph{\NomeBloco{}} funciona como uma chave, permitindo ou não o sinal de um lado passar ao outro. O bloco apresenta as definições de sinais de entrada e sa\'ida referidos na \autoref{\NomeSTab}.

\begin{table}[htbp]
\caption{Sinais do bloco \emph{\NomeBloco}}
\label{\NomeSTab}
\centering
\begin{tabular}{ccl}

    \toprule
    Sinal & Tipo    & Descrição        \\
    \midrule \midrule
    A & Bidirecional & Sinal bidirecional 1\\
    \midrule
    B & Bidirecional & Sinal bidirecional 2\\
    \midrule
    ENABLE & Entrada & Sinal de habilitação\\
    \bottomrule
\end{tabular}
\legend{Fonte: Produzido pelo autor}
\end{table}

O circuito projetado para o bloco \'e demonstrado na \autoref{\NomePFig}.

\begin{figure}[htb]
 \centering
  \begin{minipage}{0.4\textwidth}
    \centering
    \caption{\label{\NomePFig}Circuito CMOS projetado para o bloco \emph{\NomeBloco}}
    \includegraphics[scale=0.3]{Circuitos/TG.png}
    \legend{Fonte: Produzido pelo autor}
  \end{minipage}
  \hfill
  \begin{minipage}{0.4\textwidth}
    \centering
    \caption{\label{\NomeSFig}Representação em bloco do \emph{\NomeBloco}}
    \includegraphics[scale=0.3]{Circuitos/TG_Simbolo.png}
    \legend{Fonte: Produzido pelo autor}
  \end{minipage}
\end{figure}

Os transistores utilizados no bloco \emph{\NomeBloco{}} apresentam os par\^ametros mostrados na \autoref{\NomeTTab}.

\begin{table}[htbp]
\caption{Transistores do Bloco \emph{\NomeBloco}}
\label{\NomeTTab}
\centering
\begin{tabular}{ccccc}
\toprule
Transistor & W ($\mu$m)  & L ($\mu$m)           & M (n° dispositivos) & S (n° dispositivos)\\
\midrule \midrule
Q1 & 0,8 & 0,18 & 1 & 1\\
\midrule
Q2 & 0,4 & 0,18 & 1 & 1\\
\bottomrule
\end{tabular}
\legend{Fonte: Produzido pelo autor}
\end{table}
\renewcommand{\NomeBloco}{Buffer}
\renewcommand{\NomePTab}{tab_\NomeBloco}
\renewcommand{\NomeSTab}{tab_\NomeBloco2}
\renewcommand{\NomePFig}{fig_\NomeBloco}
\renewcommand{\NomeSFig}{fig_\NomeBloco2}
\renewcommand{\NomeTTab}{tab_\NomeBloco3}

\section{Buffer}
\label{buffer}

O bloco \NomeBloco{}\footnote{Circuito disponibilizado por Dalton Martini Colombo, orientador do trabalho aqui apresentado} tem a finalidade de colocar o mesmo sinal de entrada na sua sa\'ida, reduzindo efeitos de carga. O sinal de entrada pode ser tanto anal\'ogico quanto digital. O bloco apresenta as definições de sinais de entrada e sa\'ida referidos na \autoref{\NomeSTab}.

\begin{table}[htb]
\caption{Sinais do bloco \NomeBloco}
\label{\NomeSTab}
\centering
\begin{tabular}{ccl}

    \toprule
    Sinal & Tipo    & Descrição        \\
    \midrule \midrule
    In    & Entrada & Sinal de Entrada \\
    \midrule
    Out   & Saída   & Sinal de Sa\'ida   \\
    \bottomrule
\end{tabular}
\legend{Fonte: Produzido pelo autor}
\end{table}

O circuito projetado para o bloco \'e demonstrado na \autoref{\NomePFig}.

\begin{figure}[htb]
 \centering
    \caption{\label{\NomePFig}Circuito CMOS projetado para o bloco \NomeBloco}
    \includegraphics[scale=0.3]{Circuitos/Buffer.png}
    \legend{Fonte: Produzido pelo autor}
\end{figure}

\begin{figure}[htb]
 \centering
    \centering
    \caption{Representação em bloco do \NomeBloco} \label{\NomeSFig2}
    \includegraphics[scale=0.3]{Circuitos/Buffer_block.png}
    \legend{Fonte: Produzido pelo autor}
\end{figure}


Os transistores utilizados no bloco \NomeBloco{} apresentam os par\^ametros mostrados na \autoref{\NomeTTab}.

\begin{table}[htb]
\caption{Transistores do Bloco \NomeBloco}
\label{\NomeTTab}
\centering
\begin{tabular}{ccccc}
\toprule
Transistor & W ($\mu$m)  & L ($\mu$m)           & M (n° dispositivos) & S (n° dispositivos)\\
\midrule \midrule
Q1 e Q6 & 0,22 & 0,54 & 1 & 1\\
\midrule
Q2 e Q7 & 0,66 & 0,54 & 1 & 1\\
\midrule
Q3 e Q8 & 1,98 & 0,54 & 1 & 1\\
\midrule
Q4 e Q9 & 5,94 & 0,54 & 1 & 1\\
\midrule
Q5 e Q10 & 17,82 & 0,54 & 1 & 1\\
\bottomrule
\end{tabular}
\legend{Fonte: Produzido pelo autor}
\end{table}
\renewcommand{\NomeBloco}{\emph{Par Diferencial}}
\renewcommand{\NomeBlocoNoIt}{Par Diferencial}
\renewcommand{\NomePTab}{tab_\NomeBlocoNoIt}
\renewcommand{\NomeSTab}{tab_\NomeBlocoNoIt2}
\renewcommand{\NomePFig}{fig_\NomeBlocoNoIt}
\renewcommand{\NomeSFig}{fig_\NomeBlocoNoIt2}
\renewcommand{\NomeTTab}{tab_\NomeBlocoNoIt3}
\renewcommand{\NomeQTab}{tab_\NomeBlocoNoIt4}

\section{Par Diferencial}

O bloco \NomeBloco{}\footnote{Circuito disponibilizado por Dalton Martini Colombo, orientador do trabalho aqui apresentado} tem a mesma função de comparar duas entradas, fazendo a diferença entre elas, e colocar o resultado vezes um ganho na saída. O bloco apresenta as definições de sinais de entrada e sa\'ida referidos na \autoref{\NomeSTab}.

\begin{table}[htbp]
\caption{Sinais do bloco \NomeBloco}
\label{\NomeSTab}
\centering
\begin{tabular}{ccl}

    \toprule
    Sinal & Tipo    & Descrição        \\
    \midrule \midrule
    Vp (+) & Entrada & Entrada positiva do Comparador\\
    \midrule
    Vn (-) & Entrada & Entrada negativa do Comparador\\
    \midrule
    Ibias & Entrada & Corrente de polarização do Comparador\\
    \midrule
    Vo & Sa\'ida & Sa\'ida do Comparador\\
    \bottomrule
\end{tabular}
\legend{Fonte: Produzido pelo autor}
\end{table}

O circuito projetado para o bloco \'e demonstrado na \autoref{\NomePFig}.

\begin{figure}[htb]
 \centering
    \centering
    \caption{\label{\NomePFig}Circuito CMOS projetado para o bloco \NomeBloco} 
    \includegraphics[scale=0.4]{Circuitos/diff_pair.png}
    \legend{Fonte: Produzido pelo autor}
\end{figure}

\begin{figure}[htb]
 \centering
    \centering
    \caption{Representação em bloco do \NomeBloco} \label{\NomeSFig}
    \includegraphics[scale=0.3]{Circuitos/diff_pair_block.png}
    \legend{Fonte: Produzido pelo autor}
\end{figure}

Os transistores utilizados no bloco \NomeBloco{} apresentam os par\^ametros mostrados na \autoref{\NomeTTab}.

\begin{table}[htbp]
\caption{Transistores do Bloco \NomeBloco}
\label{\NomeTTab}
\centering
\begin{tabular}{ccccc}
\toprule
Transistor & W ($\mu$m)  & L ($\mu$m)           & M (n° dispositivos) & S (n° dispositivos)\\
\midrule \midrule
Q1 e Q2 & 40 & 7 & 2 & 1\\
\midrule
Q3 e Q4 & 35 & 6 & 2 & 1\\
\midrule
Q5 e Q6¹ & 70 & 4 & 1 & 1\\

\bottomrule
\end{tabular}
\legend{Fonte: Produzido pelo autor}
\legend{$^1$Calculado de forma a produzir uma corrente de 50 $\mu$A}
\end{table}

Um segundo circuito, chamado de \emph{diff\_3} de mesma topologia apresentado na \autoref{\NomePFig} foi desenvolvido, por\'em com par\^ametros distintos, dados na \autoref{pard_diff3}.

\begin{figure}[htb]
 \centering
    \centering
    \caption{Representação em bloco do diff\_3} 
    \includegraphics[scale=0.3]{Circuitos/diff_pair_3_block.png}
    \legend{Fonte: Produzido pelo autor}
\end{figure}

\begin{table}[htbp]
\caption{Transistores do Bloco diff\_pair\_3}
\label{pard_diff3}
\centering
\begin{tabular}{ccccc}
\toprule
Transistor & W ($\mu$m)  & L ($\mu$m)           & M (n° dispositivos) & S (n° dispositivos)\\
\midrule \midrule
Q1 e Q2 & 10 & 0.5 & 2 & 1\\
\midrule
Q3 e Q4 & 5 & 0.5 & 2 & 1\\
\midrule
Q5 e Q6¹ & 50 & 3 & 2 & 1\\

\bottomrule
\end{tabular}
\legend{Fonte: Produzido pelo autor}
\legend{$^1$Calculado de forma a produzir uma corrente de 50 $\mu$A}
\end{table}

\chapter{Espelho de Corrente}
\label{anexoespelhos}

\section{Espelho de Corrente NMOS}

Um espelho de corrente \'e um circuito que replica o sinal de uma corrente de refer\^encia em outras sa\'idas, podendo ser multiplicado por um fator de ajuste. A \autoref{fig_espelho} mostra a representa ção de um circuito com tal caracter\'istica, utilizando transistores NMOS. O espelho de corrente NMOS tamb\'em \'e chamado de dreno de corrente, por drenar a corrente nos ramos.

\begin{figure}[htb]
    \label{fig_espelho}
    \centering
    \caption{Espelho de corrente NMOS} 
    \includegraphics[scale=0.4]{Circuitos/current_mirror_example.png}
    \legend{Fonte: Produzido pelo autor}
\end{figure}

Nesta figura, est\'a representado a utiliza ção de uma corrente de refer\^encia Iref para gerar as correntes Io1, Io2 at\'e Ion, onde "n" \'e o n\'umero de transistores se referenciando por $Q_{ref}$.

No circuito demonstrado pela \autoref{fig_espelho}, o transistor \emph{Qref} tem funç\^ao de captar a corrente igual \'a \emph{Iref}, para servir de refer\^encia aos outros transistores \emph{Q1}, \emph{Q2} at\'e \emph{Qn}, onde \emph{n} \'e o numero de transistores que se referenciam de Q\_ref, e que são chamados de bra ços do espelho. O potencial \emph{VDD\_IREF} \'e um valor de tensão utilizado para polarizar Iref, que não precisa ser igual ao valor de alimenta ção dos outros bra ços.

Dado o parâmetro $W_x/L_x$ de cada transistor, onde "x" é o número indicado do transistor, podemos calcular o valor de corrente de cada braço utilizando a fórmula apresentada na \autoref{eq_espcorpmos}.

\begin{equation}
    \label{eq_espcor}
    I_{OX} = Iref\frac{W_x/L_x}{W_{ref}/L_{ref}}
\end{equation}

Onde $I{ox}$ \'e a corrente de entrada do bra ço \emph{Qx}. 

Para que esse circuito funcione devidamente em cada sa\'ida, cada bra ço do espelho deve estar necessariamente operando na região ativa, pois a \autoref{eq_espcor} \'e deduzida levando isso em conta. Para que isso aconte ça, devemos respeitar a \autoref{eq_curmirror_req} \cite{RazaviFundM}.

\begin{equation}
    \label{eq_curmirror_req}
    v_{DS} \geq v_{GS} - V_t
\end{equation}

Onde:

\begin{itemize}
    \item $v_{DS}$ \'e a tensão entre o dreno e fonte do transistor
    \item $v_{GS}$ \'e a tensão entre o porta e fonte do transistor
    \item $V_{t}$ \'e a tensão de limiar do transistor
\end{itemize}

\section{Espelho de Corrente PMOS}

Um circuito de espelho de corrente pode ser constru\'ido com transistores PMOS, utilizando o mesmo racioc\'inio de constru ção do circuito NMOS, conforme a \autoref{fig_curmir_pmos}. A diferen ça principal \'e que em vez de ser um dreno de corrente, o PMOS ser\'a um fornecedor de corrente.

\begin{figure}[htb]
    \label{fig_curmir_pmos}
    \centering
    \caption{Espelho de corrente PMOS} 
    \includegraphics[scale=0.4]{Circuitos/current_mirror_example_pmos.png}
    \legend{Fonte: Produzido pelo autor}
\end{figure}

O funcionamento do circuito PMOS segue o mesmo principio do NMOS, porém, ao inv\'es de receber corrente fixada a um n\'o, ele fornece. O potencial \emph{negVDD\_{IREF}} \'e um valor de tensão utilizado para polarizar Iref, que não precisa ser igual ao valor de alimenta ção negativa/terra dos outros bra ços.

Dado o parâmetro $W_x/L_x$ de cada transistor, onde "x" é o número indicado do transistor, podemos calcular o valor de corrente de cada braço utilizando a fórmula apresentada na \autoref{eq_espcorpmos}.

\begin{equation}
    \label{eq_espcorpmos}
    I_{OX} = I_{ref}\frac{W_x/L_x}{W_{ref}/L_{ref}}
\end{equation}

Onde $I_{OX}$ \'e a corrente de sa\'ida do bra ço \emph{Qx}. 

Assim como no caso do NMOS, todos transistores devem estar na região ativa, e respeitar a seguinte \autoref{eq_curmirror_reqpmos} \cite{RazaviFundM}.

\begin{equation}
    \label{eq_curmirror_reqpmos}
    v_{SD} \geq v_{SG} - |V_t|
\end{equation}

Onde:

\begin{itemize}
    \item $v_{SD}$ \'e a tensão entre o fonte e dreno do transistor
    \item $v_{SG}$ \'e a tensão entre a fonte e oirta do transistor
    \item $V_{t}$ \'e a tensão de limiar do transistor
\end{itemize}



\chapter{Encapsulamento CLCC44}
\label{anexo_clcc44}
\begin{figure}[htbp]
 \centering
    \caption{Encapsulamento CLCC44} 
    \includegraphics[scale=0.9]{Anexos/CLC44.png}
    \legend{Fonte: KYOCERA}
\end{figure}

\end{anexosenv}

% ----------------------------------------------------------
% Glossário
% ----------------------------------------------------------
%
% Consulte o manual da classe abntex2 para orientações sobre o glossário.
%
\glossary
%---------------------------------------------------------------------
% INDICE REMISSIVO
%---------------------------------------------------------------------
\phantompart
\printindex
%---------------------------------------------------------------------

