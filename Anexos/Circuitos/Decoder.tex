\renewcommand{\NomeBloco}{\emph{Decoder 2x4}}
\renewcommand{\NomeBlocoNoIt}{Decoder 2x4}
\renewcommand{\NomePTab}{tab_\NomeBlocoNoIt}
\renewcommand{\NomeSTab}{tab_\NomeBlocoNoIt2}
\renewcommand{\NomePFig}{fig_\NomeBlocoNoIt}
\renewcommand{\NomeSFig}{fig_\NomeBlocoNoIt2}
\renewcommand{\NomeTTab}{tab_\NomeBlocoNoIt3}

\section{Decoder 2x4}

O bloco \NomeBloco{}\footnote{Circuito disponibilizado por Dalton Martini Colombo, orientador do trabalho aqui apresentado} tem a função de colocar um bit '1' em uma das sa\'idas, enquanto todas outras são iguais a '0'. A \autoref{\NomePTab} indica a Tabela Verdade do bloco.

\begin{table}[htbp]

\caption{Tabela Verdade do bloco \NomeBloco}%
\label{\NomePTab}
\centering
\begin{tabular}{cccccc}
    \toprule
    In0 & In1 & Out0 & Out1 & Out2 & Out3 \\
    \midrule \midrule
    0 & 0 & 1 & 0 & 0 & 0 \\
    \midrule
    0 & 1 & 0 & 1 & 0 & 0 \\
    \midrule
    1 & 0 & 0 & 0 & 1 & 0 \\
    \midrule
    1 & 1 & 0 & 0 & 0 & 1 \\
\bottomrule

\end{tabular}
\fonte{Produzido pelo autor.}
\end{table}

O bloco apresenta as definições de sinais de entrada e sa\'ida referidos na \autoref{\NomeSTab}.

\begin{table}[htbp]
\caption{Sinais do bloco \NomeBloco}
\label{\NomeSTab}
\centering
\begin{tabular}{ccl}

    \toprule
    Sinal & Tipo    & Descrição        \\
    \midrule \midrule
    In1    & Entrada & Primeira entrada de seleção de sa\'ida \\
    \midrule
    In2    & Entrada & Segunda entrada de seleção de sa\'ida \\
    \midrule
    Out0 & Sa\'ida & Sa\'ida 0\\
    \midrule
    Out1 & Sa\'ida & Sa\'ida 1\\
    \midrule
    Out2 & Sa\'ida & Sa\'ida 2\\
    \midrule
    Out3 & Sa\'ida & Sa\'ida 3\\
    \bottomrule
\end{tabular}
\legend{Fonte: Produzido pelo autor}
\end{table}

O circuito projetado para o bloco \'e demonstrado na \autoref{\NomePFig}.

\begin{figure}[htbp]
 \centering
    \centering
    \caption{\label{\NomePFig}Circuito CMOS projetado para o bloco \NomeBloco}
    \includegraphics[scale=0.3]{Circuitos/decoder.png}
    \legend{Fonte: Produzido pelo autor}
\end{figure}

\begin{figure}[htbp]
 \centering
    \centering
    \caption{\label{\NomeSFig}Representação em bloco do \NomeBloco} 
    \includegraphics[scale=0.3]{Circuitos/decoder_block.png}
    \legend{Fonte: Produzido pelo autor}
\end{figure}