

\section{\textit{Domínio do problema}}\label{sec:Cap2_MR}

Pilotagem tripulada consiste, além da leitura dos instrumentos, em referências visuais, como ponto de referencias, terrenos, e marcações. Apesar dos instrumentos de leitura, a pilotagem tripulada é mais eficiente quando se utiliza de referências visuais, por se tratar de uma alternativa às falhas dos instrumentos. A navegação visual é uma tarefa complexa de ser implementada autonomamente, pois requer identificação de características do terreno em diversas condições e contextos. Em~\cite{COUTURIER2021103666} é apresentado duas formas abordagens de navegação visual. A primeira citada é a localização visual relativa (RLV), que tem como objetivo estimar a posição baseado no deslocamento de um frame para o outro. A segunda abordagem é a localização visual absoluta, que busca comparar as imagens do VANT ou aeronave com imagens georreferenciadas. A segunda alternativa é interessante por se tratar de imagens rotuladas com as referencias geográficas e por serem referencias absolutas, que não sofrem de drift em relação a um referencial relativo. Para implementar a AVL, uma das possíveis abordagens é utilizando CNNs. Mesmo que os custos computacionais para treinar uma rede sejam altos, a utilização do modelo em um VANT pode ter custos reduzidos, que é um aspecto vantajoso, já que estes possuem recursos computacionais reduzidos.

Em~\cite{COUTURIER2021103666} também são citadas soluções que envolvem template matching. Contudo essa solução envolve um alto custo para se realizar uma busca em uma área grande, sendo portanto necessário a implementação de técnicas como janelas deslizantes ou segmentação semântica de terreno, como foi proposto em~\cite{9552597}. 
\section{\textit{Definição de CNNs}}\label{sec:Cap2_MR}

O termo redes neurais profundas, ou deep learning, se refere a redes neurais artificiais com múltiplas camadas ocultas.Foram uma das principais tecnologias de aprendizado de máquina desenvolvidas nos últimos anos, e se tornaram cada vez mais popular. Devido a sua superior performance em extração de características, teve sucesso por distintos domínios, como visão computacional, reconhecimento de fala, processamento natural de linguagem e em big data. Uma arquitetura clássica é a rede neural convolucional (CNN), que utiliza convoluções para extrair características de uma imagem entre cada camada de filtros. Também possui camadas de pooling, não lineares e camadas completamente conectadas~\cite{8308186}. Uma dos pressupostos das CNNs é que os filtros são indiferentes a translações das características na imagem, possibilitando assim uma eficiente extração de características para composição e identificação da imagem.

\section{\textit{O problema de rotulagem e variabilidade de amostras de treino}}\label{sec:Cap2_MR}


Outro desafio envolvido no treino de CNNs é representar um estado de características que cubram as variações fotográficas, tanto em características do sensor, como variações da imagem no dia, clima, estação e plataforma da camera se torna um desafio difícil. Para uma localização efetiva, o modelo deve ser robusto a todas essas características, que requer um grande conjunto de treino que cobre boa parte das diversas condições possíveis. Tal conjunto de dados não é disponível e nem viável de obter, pois seria um volume muito grande de~\cite{rs13194017}. Também não é viável o aprendizado contínuo em tempo real, dado o alto custo de poder computacional e tempo. Tais limitações levam a necessitar o desenvolvimento de algorítimos que aprendem seletivamente para que o podem computacional seja utilizado de forma eficiente, bem como reutilizar conhecimento prévio e evitar treinamento redundante~\cite{rostami2019learning}.  Dentre as técnicas utilizadas para implementar esses modelos mais eficientes, temos como exemplo o aprendizado supervisionado fraco, ou transferência de aprendizado. 

\section{\textit{Aprendizado semi-supervisionado}}\label{sec:Cap2_MR}

As técnicas de aprendizado semi-supervisionado consistem em treinar um modelo com apenas um conjunto de amostras de treino com suas respectivos rótulos. As demais amostras podem por exemplo, serem agrupadas e rotuladas como a amostra mais próxima, como apresenta o trabalho de~\cite{Sanches2003}

\section{\textit{Transfer learning}}\label{sec:Cap2_MR}

Já técnicas de transferência de aprendizado, ou few shots learning, consistem em redes treinadas para um conjunto limitado de testes~\cite{rostami2019learning}

\section{\textit{Topologias de CNN propostas na literatura}}\label{sec:Cap2_MR}


topologias de dnn  modelos bons de dnn pra localização 





\section{\textit{Compressão de DNNs}}\label{sec:Cap2_MR}


Redes neurais profundas estão no estado da arte em aplicações de visão computacional, à nível de capacidade humana, contudo ao custo da faixa de dezenas a centenas de milhões de parâmetros e complexidade computacional. Também são dependem muito tempo e energia para treinar, e vasto conjunto de dados de treino. Essas grandes DNNs são difíceis de serem implementadas em ambientes embarcados, devido a fatores limitantes como banda, instruções por segundo e memória. 

Para lidar com problema de limitações de recursos, uma abordagem possível é a técnica de podagem de redes neurais profundas, que remove as neurônios que não são relevantes para a rede. Como é citado em~\cite{jordao2019pruning}, foi possível reduzir até 67\% das operações de ponto flutuante (FLOPs), sem perda na acurácia e também foi possível reduzir 90\% do custo com uma perda negligente. Neste trabalho foi possível melhorar a acurácia comparado ao original, devido a regularização da rede. Consiste em estimar a importância do neurônio baseado em sua relação com a classe em um espaço dimensional reduzido. A relação é computada utilizando mínimos quadrados parciais e importância de variável em projeção. Em~\cite{jordao2019pruning} também são mencionadas outras importante técnicas de compressão de DNNs, como quantização de float, que consiste da redução de float de 32 bits pata 8 bits, nos pesos e funções de ativação, reduzindo a precisão, acelerando as operações atômicas, contudo sem impacto relevante na acurácia do modelo. Aceleração de hardware também possibilita operações atômicas de multiplicações matriciais e altamente paralelizadas. 

\section{\textit{Trabalhos relacionados}}\label{sec:Cap2_MR}


Em \cite{jeon2021run} foram feitos benchmarks de possíveis hardwares para implementações de odometria visual utilizando cameras stereo e fusões sensoriais. Neste trabalho é possível concluir a alto ganho de desempenho devido a aceleração de hardware dos módulos Nvidia Jetson, que utilizam unidades gráficas para acelerar o treino das CNNs. E atendendo os requisitos de potência, consumindo apenas 7.5W e e de tamanho, podendo ser portada em micro aeronaves.  Em [] e [] também foi realizada a implementação de localização visual, contudo utilizando Raspberry pi, que por sua vez possui mais recursos de processamento na CPU do que GPU. Contudo  o que não garantiu desempenho suficiente para uma boa acurácia na localização visual. Portanto, para se portar um modelo em uma micro aeronave, dado as limitações de peso e potência e acurácia mínima é interessante utilizar sistemas embarcados que possuem aceleração de hardware, especializada em computação gráfica, como os módulos Jetson da NVIDIA.
