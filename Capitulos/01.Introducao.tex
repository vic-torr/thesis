O número de casos de usos de veículos aéreos não tripulados (VANTs), inicialmente predominante em aplicações militares, tem crescido rapidamente aplicações civis como: segurança, agricultura de precisão, inspeção de infraestrutura, sensoriamento remoto e entregas de bens. Possuindo grande vantagens às alternativas comerciais já estabelecidas em termos de redução de riscos e custos~\cite{8682048}. VANTs operando em maiores distancias, também classificados como operação além do campo de visão, podem requirir operações autônomas ou semi-autônomas, em que o operador apenas intervem dando comandos de auto-nível ao sistema de navegação. Para uma operação autônoma adequada, o sistema de navegação, por sua vez, é crítico possuir um sistema robusto e confiável de auto-localização~\cite{COUTURIER2021103666}.

A localização consiste em estimar a pose do veículo, podendo ser a pose representada em em um vetor de 6 graus de liberdade (x,y,z,\(phi\), \(theta\), \(Phi\)), bem como representações simplificada de 2 graus de liberdade (x,y).
As implementações mais amplamente utilizadas envolvem o sistema de posicionamento global (GPS). Implementações mais robustas também combinam as medições do GPS com da unidade inercial (IMU) do veículo para estimar a pose com mais acurácia. Essa combinação é realizada a partir de fusão sensorial utilizando Filtros Gaussianos, como o filtro de Kalman. Contudo, é relevante enfatizar que sozinhos cada uma das medições são limitadas: O GPS possui baixa acurácia e taxa de atualização, enquanto estimar pose pela IMU gera alto erro de drift~\cite{COUTURIER2021103666}.

Dessa forma, temos que tais sistemas de localização são altamente dependentes de GPS. E isso é uma limitação, pois o GPS não é sempre disponível em todos ambientes, principalmente por eventos como interferência destrutivas for reflexão de sinal, bloqueio por obstáculo, por condição climática, negação deliberada de sinal (signal jamming) e falsificação do sinal (signal spoofing).
O primeiro acontece em ambientes onde existem superfícies metálicas que refletem o sinal sinal originado dos satélites, de forma que o receptor recebe oe mesmo sinal por múltiplos caminhos, frequente em ambientes urbanos ou fechados. O segundo acontece quando o sinal é propositalmente bloqueado por meio de uma transmissão de mesma frequência. O terceiro, potencialmente mais perigoso, é a falsificação do sinal, de forma que o módulo GPS interpreta erroneamente a pose~\cite{6837385}. Esses três eventos são exemplos de ambiente onde há a negação de GPS e que comprometem o sistema de navegação, impossibilitando a navegação autônoma.

Diante dessa limitação, de situações de com negação de GPS, surge a necessidade de alternativas de segurança de localização. Dentre as possíveis, temos como exemplo sistemas baseados em estimação de posição por localização visual, que consiste na aeronave usar cameras que capturam imagens do solo para localizar e navegar, sejam imagens do espectro visíveis ou imagens de amplo espectro. Neste trabalho, nos concerne estudar essa opção, onde são pesquisadas técnicas como a odometria visual (VO), localização e mapeamento simultâneo (SLAM), localização absoluta (AVL) por imagens georreferenciadas, seja por template matching, filtro de partículas, ou redes neurais convolucionais.

No que tange a sistemas de visão computacional, é relevante mencionar o progresso nos últimos anos providos por redes neurais profundas e redes neurais convolucionais (CNNs), que atingiram o estado da arte em numerosas tarefas. Jà são amplamente aplicadas com sucesso nos campos da robóticas e de visão computacional, porém ainda limitadas em aplicações de UAV, devido a restritiva limitação de recursos de sistemas embarcados de aeronaves de pequeno porte e ao ao alto custo computacional de redes neurais profundas. Outro ponto relevante de mencionar é o custo alto de treinos supervisionados e a necessidade de uma elevada quantidade de amostras de treino rotuladas, que podem dificultar a elaboração de um modelo de boa acurácia.

Dado todos esses fatores, este trabalho busca estudar alternativas de localização visual sob todas as limitações citadas, tangendo limitações de hardware e modelos com poucos dados rotulados. Bem como propor uma implementação factível. Consiste de mais quatro capítulos, respectivamente: revisão bibliográfica, metodologia para implementação de solução, implementação, e validação e análise de resultados. Concluindo com contribuições e direções futuras para este problema estudado.
