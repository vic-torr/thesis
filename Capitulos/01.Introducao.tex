A floresta amazônica sofre uma degradação histórica, com perda de até 19\% em área de vegetação desde os anos 1970, atingindo uma perda de \( 10.000~km^2 \) em 2020 \cite{ignacioNature}. Há previsões que ao atingir um limiar de 20\% a 25\% de perda do bioma, acarretará um ponto de inflexão, onde a floresta deixará de ser auto sustentável e será substancialmente transformada, se tornando mais infértil e árida~\cite{LovejoyTipping}. Se deve ao fato do ecossistema possuir alta densidade de vegetação e  taxa de evapotranspiração, formando os chamados rios aéreos, que sustentam as monções e alta taxa de precipitação~\cite{satyamurty2013moisture}. Ao se remover a vegetação desse ciclo, o volume de  precipitação pode decair dramaticamente. Já existem evidências~\cite{ignacioNature} de vegetação adaptada ao cerrado e climas mais áridos predominando na região leste da amazônia, território que sofreu grandes perdas de vegetação. Predições para tal ponto de inflexão apontam que até 2050 cerca de metade da floresta em território brasileiro não sobreviverá~\cite{LovejoyTipping}.

Tais evidências demonstram o quão urgente são necessárias medidas para monitorar e coibir práticas ilegais de desflorestamento. Atualmente estas áreas podem ser detectadas remotamente pelo MapBiomas, o sistema de monitoramento de tempo real do Instituto Nacional de Pesquisa Espacial, o INPE. São realizadas múltiplas varreduras em amplo espectro, desde o micro-ondas, infravermelho ao espectro visível. Aplicando técnicas de fusão sensorial, é possível identificar as regiões onde tais práticas acontecem extensivamente~\cite{inpe_deter}. Dentre elas, queimadas, desmatamento extrativo, garimpo, agricultura e agropecuária irregulares.
\begin{figure}[h!]
 \centering
 \includegraphics[width=0.9\columnwidth]{Imagens/Ti-Munduruku-Foto_Marizilda_Cruppe_Amazônia_Real.jpg}
 \caption{Garimpo ilegal na Terra Indígena Munduruku, município de Jacareacanga.
 Foto: Marizilda Cruppe/Amazônia Real}
\label{fig:garimpo}
\end{figure}
Contudo, soluções automatizadas atualmente implementadas possuem baixa resolução, de áreas de \(250\times250m\). O que significa que explorações de menores escalas ou esparsas ainda podem ser difíceis de serem detectadas~\cite{mapbiomasgarimpo}. Temos ainda que o INPE, conta atualmente com satélite CBERS-4 que possuem sensores na faixa do espectro visível e de maior resolução espacial. Isso permite realizar a detecção de ocupações irregulares ainda em fase iniciais, bem como identificar práticas que são menos extensivas em área, embora ainda muito prejudiciais. Como exemplo, temos o recente caso de de explosões de inúmeras áreas de garimpo do rio madeira, que passariam desapercebido das detecções de maiores resoluções~\cite{mapbiomasgarimpo}.


O garimpo também é uma das causas raízes da degradação do bioma, já que a Amazônia concentra 94\% (mais de 100 mil hectares) da área garimpada brasileira, sendo mais de 50\% potencialmente ilegais, por ocorrerem dentro em Terra Indígenas (TIs) e Unidades de Conservação (UCs). A área de garimpo no bioma cresceu 10x nas últimas três décadas, com 301\% de expansão em UCs e 495\% em TIs~\cite{mapbiomasgarimpo}.


Uma possível solução para detecção de tais irregularidades são modelos de visão computacional e aprendizado profundo para classificação automatizada de sub regiões na faixa de captura dos satélites. Tais sistemas seriam treinados com amostras de regiões regulares de mata nativa e regiões onde ocorrem irregularidades, para serem capazes de classificar e diferenciar cada categoria.

Uma das técnicas que se destacaram, na última década, pela atuação em visão computacional foram as redes neurais convolucionais. Conseguiram um salto de precisão ao resolver a competição de identificação de imagens ImageNet, através de modelos como AlexNet e ResNet~\cite{alom2018history}. Contudo, para aplicações de imagens de satélite, também chamada sensoriamento remoto, podem não possuir o mesmo desempenho em relação a objetos do cotidiano~\cite{wang2022empirical}. Isso se deve a alta variabilidade entre amostras dentro de uma classe a ser identificada e similaridades com amostras fora da classe.

Outra adicional dificuldade enfrentada por redes convolucionais em sensoriamento remoto é o grande volume de dado necessário para treinar o modelo, sob diferentes condições, como imagens rotacionadas, diferentes sensores, luminosidade no momento da captura, condições climáticas e nuvens. Tais variabilidades limitam a robustez e demandam um conjunto de treino que represente estatisticamente as possíveis diferentes condições~\cite{5782957}.

Já uma recente família de modelos, chamados Transformers Visuais, tem ganhado espaço no campo de visão computacional nos últimos anos. Foram capazes e de superar o desempenho de modelos baseados em CNN, em algumas aplicações, utilizando menos pesos e sendo mais barato computacionalmente~\cite{wang2022empirical}. Isto pode ser atribuído à chamada propriedade “atenção”, onde o contexto de cada parte de entrada, em relação às demais partes, tem peso para classificação~\cite{dosovitskiy2020image}. Dessa forma, tais classificadores têm potencial de terem maior robustez em relação às variações de entradas.


Dado todos esses fatores, este trabalho terá como objetivo classificar regiões irregulares, bem como comparar com modelos previamente implementados. Também investigará alternativas de detecção visual aplicadas a sensoriamento remoto, utilizando modelos de Transformers Visuais pré-treinados com imagens de satélite de diferentes condições. Contemplará as limitações de quantidade reduzida de amostras de treino e desbalanceada para a quantidade de classes de interesse. Assim como realizar uma implementação factível. Consiste em mais quatro capítulos, respectivamente: revisão bibliográfica, metodologia para a implementação de solução, análise de resultados e a conclusão, incluindo contribuições e direções futuras para este problema estudado.




