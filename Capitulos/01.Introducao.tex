
Sistemas aviônicos de navegação são altamente dependentes de localização baseada em GPS. Especialmente os
veiculos aéreos não tripulados (VANTs), que são ou pilotados ou autonomos. Podem operar Possuem crescente mercado e aplicações nas indústrias agrícolas,
militares e serviços, onde agregam o propósito de inteligência e reconhecimento 
de terreno . E para que sua navegação seja automatizada e
realizável, como é no caso dos VANT \-- veículos autônomos aéreos não tripulados
\-- é imprescindível a implementação de um sistema de localização robusto.

As formas de localização mais amplamente utilizadas são a partir de GPS podendo
ser combinados ou não com fusão sensorial por meio de um filtro Gaussiano, que
combina com as medições de uma IMU \-- Unidade de medição inercial, para
medições mais precisas e rápidas. O GPS embora tenha acurácia da ordem de poucos
metros, possui o compromisso de ter baixa taxa de amostragem. alguns
compromissos consideráveis. Isso pode ser mitigado com a fusão sensorial com
medições da IMU, assim obtendo melhores estimativas de posição. A estimação de
posição utilizando somente a IMU também pode não ser adequada pelo fato da
estimativa ser calculada integrando as medições de aceleração e assim possuindo
alto drift. Dessa forma, essa estimação clássica de posição é mais adequada pelo
conjunto de ambos. Este conjunto, contudo, deixa de ser preciso quando não há a
disponibilidade de GPS. Essa condição pode ocorrer por eventuais
indisponibilidades, também por interferências destrutivas comuns em um ambiente
urbano ou geradas intencionalmente. Ambientes urbanos possuem superfícies
metálicas que refletem o sinal transmitido pelos satélite. Tal sinal refletido
ao chegar no receptor GPS pode gerar interferências destrutivas com o sinal
original e o refletido. Signal spoofing é uma vulnerabilidade existe para VANTs,
seja de aplicações civis ou militares, já que um agente malicioso pode
desabilitar a localização ou comunicação do VANT por meio de um sinal de
interferência.

Outro aspecto característico de VANT é que são sistemas computacionais embarcados com limitações de capacidade de hardware e energia. O que significa que o modelo deve ser otimizado para não consumir recursos computacionais ilimitados.

Para que a localização seja realizada, é necessário que o VANT possua um sistema de localização robusto. E neste trabalho serão comparadas técnicas de otimização a fim de implementar modelos de trabalhos prévios.  proposto e implementado um sistema de localização via pattern matching complementar às localizações anteriormente citadas. O reconhecimento de padrões será feito a partir da comparação das imagens de georreferenciadas de satélite com as imagens capturadas instantaneamente. Além disso, terá como escopo portar o modelo de reconhecimento de padrões para o hardware limitado do VANT e metodologias de reduzir o custo computacional de avaliação de padrões.


Todo trabalho será desenvolvido com apoio do laboratório MINDS (Machine Intelligence and Data Science lab) da Universidade Federal de Minas Gerais interno a Escola de Engenharia, que possui o ferramental e orientação necessários.


Este trabalho abrange quatro capítulos além deste. O segundo capítulo contém a revisão bibliográfica para o estudo de caso proposto. O terceiro capítulo aborda a metodologia adotada para a realização da proposta e o desenvolvimento do software. Adiante, o quarto capítulo, contém a análise dos resultados esperados. Por fim, o capítulo cinco conclui este trabalho e aponta direções futuras para esta área de pesquisa.


