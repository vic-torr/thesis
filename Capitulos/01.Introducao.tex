
VANT ou veiculos autonomos aéreos não tripulados possuem crescente potencial e  aplicação nas indústrias agricolas,
militares e serviços. E para que sua navegação seja realizável, é imprescindível a implementação de um sistema de
localização robusto [citar business report]. As formas de localização mais amplamente utilizadas são a partir de GPS
podendo ser combinados ou não com fusão sensorial à uma IMU \-- Unidade de medição inercial. O GPS embora tenha
acurácia da ordem de poucos metros, possui o compromisso de ter baixa taixa de amostragem. Isso pode ser mitigado com a
fusão sensorial com medições da IMU, assim obtendo melhores estimativas de posição. A estimação de posição utilizando somente a IMU também pode não ser adequada pelo fato da estimativa ser calculada integrando as medições de aceleração e assim possuindo alto drift. Dessa forma, essa estimação clássica de posição é mais adequada pelo conjunto de ambos.
Este conjunto, contudo, deixa de ser preciso quando não há a disponibilidade de GPS. Essas condição pode ocorrer por
eventuais indisponibilidades, também por interferencias destrutivasm comuns em um ambiente urbando ou por signal jamming
deliberadamente. Ambientes urbanos possuem superfícies metálicas que refletem o sinal transmitido pelos satélite. Tal
sinal refletido ao chegar no receptor GPS pode gerar interferências destrutivas com o sinal original desejado. Signal jamming é uma vulnerabilidade comum para drones, seja de aplicações civis ou militares, já que um agente maliciosoo pode desabilitar a localização ou comunicação do drone por meio de um sinal de interferência. 


Para que a localização seja realizada, é necessário que o drone possua um sistema de localização robusto. E neste
trabalho será proposto e implementado um sistema de localização via pattern matching complementar as localizações
anteriormente citadas. O patern matching será feito a partir da comparação das imagens de georeferendiadas de satélite com as imagens captturadas instantaneamente. Além disso, terá como escopo portar o modelo de reconhecimento de padrões para o hardware limitado do drone e metodologias de reduzir o custo computacional de avaliação de padrões.


Todo trabalho será desenvolvido com apoio do laboratório MINDS (Machine Intelligence and Data Science) da Universidade Federal de Minas Gerais interno a Escola de Engenharia, que possui o ferramental e orientação.

A necessidade de atribuir a dispositivos tarefas que fazemos no dia a dia se tornou algo comum na atualidade. Isso se deve a uma evolução que se mostrou evidente tanto em termos de hardware quanto de software, aparecendo produtos no mercado que atendem tanto a demanda industrial quanto a comercial. E, na medida em que certa tecnologia cresce e tem o aumento da comunidade desenvolvedora trabalhando em conjunto, ela acaba se tornando cada vez mais acessível, possibilitando uma maior interação interdisciplinar.

Dentro deste contexto, pode-se observar a crescente aplicação de manipuladores robóticos em setores diversificados. O uso mais evidente é dentro do contexto industrial, uma vez que os manipuladores robóticos podem automatizar processos antes feitos manualmente pelos funcionários, podendo assim oferecer agilidade, segurança, dentre outros benefícios às empresas. Porém, pode-se observar também que estes estão sendo utilizados cada vez mais comercialmente, como é o caso de \textit{gimbals}, \textit{hoverboards} e \textit{drones}.

Ainda na vertente do crescimento tecnológico, os equipamentos atualmente possuem hardware potentes a preços acessíveis, capazes de rodar aplicações complexas que vão de encontro a popularização de ideias, como a Internet das Coisas. Os dispositivos permitem cada vez mais essa integração, tanto com aplicações industriais quanto domésticas. Essa popularização permitiu que várias bibliotecas fossem criadas e publicadas para a comunidade desenvolvedora a fim de instigar mais o desenvolvimento de possiblidades criativas para o mercado.

Graças a performance dos dispositivos atuais, houve um aumento de aplicações que exigem uma maior complexidade no processamento de dados. Um exemplo que vem crescendo nos últimos anos é o uso de reconhecimento de padrões, a qual vem sendo utilizado em meios de produção e até mesmo como meio de entretenimento em redes sociais com o uso de filtros digitais. A finalidade do presente trabalho é desenvolver um software que utilize de ferramentas de reconhecimento de padrões a fim de entregar parâmetros para controlar um manipulador robótico.

Como a maior parte da tecnologia é lançada com o objetivo de entregar mais conforto aos usuários, muitas fabricantes têm apostado em dispositivos que utilizam da tecnologia \textit{hands-free}. Ela possibilita que os usuários desfrutem de seus produtos sem a necessidade de tocá-los, sendo a maior aposta nos comandos de voz.

Com o maior acesso a tecnologia, houve um crescente uso de aplicações com câmera. Isso foi acentuado durante a pandemia de COVID-19, elevando o número de chamadas de vídeo realizadas ao redor do mundo devido às restrições sanitárias. Com intuito de integrar a tecnologia \textit{hands-free} na utilização de câmeras em dispositivos móveis, o software desenvolvido neste projeto tem o objetivo de enquadrar as pessoas na tela, permitindo que se movimentem e realizem outras tarefas durante a captura de imagem. Para isso, o dispositivo será instalado no manipulador robótico que, a partir do uso de reconhecimento facial, recebe comandos de controle para posicionar a câmera de forma adequada.

Este trabalho abrange quatro capítulos além deste. O segundo capítulo contém a revisão bibliográfica para o estudo de caso proposto. O terceiro capítulo aborda a metodologia adotada para a realização da proposta e o desenvolvimento do software. Adiante, o quarto capítulo, contém a análise dos resultados esperados. Por fim, o capítulo cinco conclui este trabalho e aponta direções futuras para esta área de pesquisa.


