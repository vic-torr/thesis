Em \cite{jeon2021run} foram feitos benchmarks de possíveis hardwares para implementações de odometria visual utilizando cameras stereo e fusões sensoriais. Neste trabalho é possível concluir a alto ganho de desempenho devido a aceleração de hardware dos módulos Nvidia Jetson, que utilizam unidades gráficas para acelerar o treino das CNNs. E atendendo os requisitos de potência, consumindo apenas 7.5W e e de tamanho, podendo ser portada em micro aeronaves.  Em [] e [] também foi realizada a implementação de localização visual, contudo utilizando Raspberry pi, que por sua vez possui mais recursos de processamento na CPU do que GPU. Contudo  o que não garantiu desempenho suficiente para uma boa acurácia na localização visual. Portanto, para se portar um modelo em uma micro aeronave, dado as limitações de peso e potência e acurácia mínima é interessante utilizar sistemas embarcados que possuem aceleração de hardware, especializada em computação gráfica, como os módulos Jetson da NVIDIA.