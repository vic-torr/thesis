Pilotagem tripulada consiste, além da leitura dos instrumentos, em referências visuais, como ponto de referencias, terrenos, e marcações. Apesar de altamente dependente dos instrumentos, a pilotagem tripulada é mais eficiente quando se utiliza de referências visuais, por se tratar de uma alternativa às falhas dos instrumentos. A navegação visual é uma tarefa complexa de ser implementada autonomamente, pois requer identificação de características do terreno em diversas condições e contextos. Em~\cite{COUTURIER2021103666} é apresentado duas formas abordagens de navegação visual. A primeira citada é a localização visual relativa (RLV), que tem como objetivo estimar a posição baseado no deslocamento de um frame para o outro. A segunda abordagem é a localização visual absoluta, que busca comparar as imagens do VANT ou aeronave com imagens georreferenciadas. A segunda alternativa é interessante por se tratar de imagens rotuladas com as referencias geográficas e por serem referencias absolutas, que não sofrem de drift em relação a um referencial relativo. Para implementar a AVL, uma das possíveis abordagens é utilizando CNNs. Mesmo que os custos computacionais para treinar uma rede sejam altos, a utilização do modelo em um VANT pode ter custos reduzidos, que é um aspecto vantajoso, já que estes possuem recursos computacionais reduzidos.