
O autor do trabalho apresentou a base teórica dos dispositivos APS e TIA, além das especificações de um projeto de Receptor Óptico, as soluções propostas pelo autor para o seu desenvolvimento, simulações para avaliação do dispositivo, e a elaboração do layout.

O projeto a nível de esquemático e o layout foi demonstrado para os principais blocos que compõem a solução proposta, com todos parâmetros de componentes especificados.

Simulações foram realizadas de forma a mostrar o funcionamento e viabilidade do dispositivo, que apresenta comportamento como esperado de toda teoria apresentada.

Com todo o trabalho aqui demonstrado, espera-se que o entendimento sobre o tema de Receptores Ópticos seja adequado para que o leitor possa replicar o projeto, adaptando-o para a aplicação que for direcionada, e que se possa realizar a validação do dispositivo de forma a demonstrar sua funcionalidade.

Desafios apareceram ao longo de todo o trabalho, tanto no entendimento da teoria por trás do tema, questões de desenvolvimento, projeto e simulação, dos quais necessitaram de ampla pesquisa acadêmica e também orientação para o seu devido entendimento e solução. Tudo demonstrado ao longo do trabalho veio de extensas tentativas do autor para possibilitar uma solução viável e que possa ser replicada. 

O trabalho como um todo teve seus objetivos alcançados, mostrando o desenvolvimento e simulação de um projeto de Receptor Óptico para três cores, e um circuito integrado foi desenvolvido para que seja realizado a sua validação.

\section{Trabalhos futuros}

O autor propõe que com a fabricação do circuito integrado, sejam realizadas todas as medições para validação do trabalho, e que este possa servir de conhecimento para outros que se proporem em estender o que foi aqui desenvolvido.

O tema pode ser expandido de forma a se validar outros Receptores Ópticos (como o \textit{CCD}), de forma a realizar uma comparação de diferentes tecnologias.